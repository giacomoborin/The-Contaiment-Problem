\documentclass[]{article}
\usepackage[T1]{fontenc}
\usepackage[utf8]{inputenc}
\usepackage[english]{babel}
\usepackage{amssymb,amsmath,amsthm,mathtools} 
\usepackage{tikz-cd,wrapfig}
\usepackage{tcolorbox}


% Theorem definitons 
\theoremstyle{plain}
\newtheorem{teo}{Theorem}[section]
\newtheorem{lem}[teo]{Lemma}
\newtheorem{prop}[teo]{Proposition}
\newtheorem{cor}[teo]{Corollary}
\newtheorem*{form}{Formula}

\theoremstyle{remark}
\newtheorem{rem}{Remark}
\newtheorem{rems}[rem]{Remarks}

\theoremstyle{definition}
\newtheorem{deff}[teo]{Definiton}
\newtheorem{idea}{Idea}
\newtheorem*{nota}{Notation}

%Bibliography
\usepackage[style=numeric, maxnames=4,backend=bibtex]{biblatex}
% other styles: numeric authortitle
\addbibresource{biblio.bib}
\usepackage{hyperref}

% Commands 
\newcommand{\Z}{\mathbb{Z}}
\newcommand{\F}{\mathbb{F}}
\newcommand{\K}{\mathbb{K}}
\newcommand{\ZZ}[1]{\mathbb{Z}_{#1}}
\newcommand{\Q}{\mathbb{Q}}
\newcommand{\C}{\mathbb{C}}
\newcommand{\R}{\mathbb{R}}

\newcommand{\p}{\mathfrak{p}}
\newcommand{\q}{\mathfrak{q}}
\newcommand{\A}{\mathfrak{a}}
\newcommand{\B}{\mathfrak{b}}
\newcommand{\Cc}{\mathfrak{c}}


\DeclareMathOperator*{\eqb }{=}
\DeclareMathOperator{\ord}{ord}
\DeclareMathOperator{\rad}{rad}
\DeclareMathOperator{\Ann}{Ann}
\DeclareMathOperator{\Ass}{Ass}



%opening
\title{The Containment Problem}
\author{Giacomo Borin}



\begin{document}

\maketitle

\begin{abstract}
	Stuff x 3
\end{abstract}

\section{Inroduction}
%TODO Should I say that R is always commutative with unity


\subsection{Associated primes}

Let $ R $ be a commutative ring with unity, and $ \A $,$ \B $ two ideal, we say that the ideal
\begin{equation*}
	(\A : \B) = \{ x \in R \,|\, x\B \subseteq \A  \}
\end{equation*}
 \nocite{AMCD}
is the \textit{ideal quotient}. For the case in which $ \A $  is the null ideal $ 0 $ we define the \textbf{annihilator} of $ \B $ as:
\begin{equation*}
	\Ann_R(\B) = (0 : \B) = \{ x \in R \,|\, x\B = 0  \}
\end{equation*}
We can obviously omit the index $ R $ if it is clear by the context. In general given an $ R  $-module $ M $ and a set $ S \subseteq M $ non empty we can define its annihilator as:
\begin{equation*}
	\Ann_R (S) = \{ x \in R \,| xS = 0  \} = \{ x \in R \,|\, \forall s \in S \; xs = 0  \}
\end{equation*}

\begin{deff}[Associated Prime]
Let $ M $ be an $ R $-module. A prime ideal $ \p \subseteq R $ is an \textbf{associated prime} of $ M $ if there exists a non-zero element $ a \in M $ such that $ \p = \Ann_R (a)$. \\
We define $ \Ass_R(M) $ as the set of the associated primes of $ M $.\\
For an ideal $ I $ we say that a prime is associated to $ I $ if it is associated to the $ R $-module $ R/I $.
\end{deff}

\begin{rem}
	Another name for associated ideal used by the Bourbaki group is \textit{assasin}, a word play between associated and annihilator. %todo check this
\end{rem}



\subsection{Primary decomposition}

%TODO Mettici qualcosa di meglio dai
We would like to have some sort of factorization for the ideals of a ring, more general than the \textit{unique factorization domains}, in fact this is useful only for principal ideals. With this objective \textbf{primary decomposition} was introduced. 

Now I will recall some of the principal result on this topics, contained in \cite[Section 7]{Reid} and \cite[Section 4 and Page 83]{AMCD}

\begin{deff}
	An ideal $ \A $ in a ring $ R $ is said primary if $ R/\A$ is different from zero and all its zerodivisors are nilpotent, otherwhise we can express this as:
	\begin{equation*}
		fg \in \aa \Longrightarrow f \in \A \text{ or } g^n \in \A \text{ for some } n >0
	\end{equation*}
\end{deff}

It is obvious that the radical of a primary ideal is a prime ideal, infact given $ fg \in \rad(\A)  $ we have $ (fg)^m = f^m g^m \in \A $ for $ m>0 $, and so $ f^m \in \A \Rightarrow f \in \rad(\A)$ or exists $ n>0 $ such that $ g^{mn} \in \A \Rightarrow g \in \rad(\A) $.

If $ \A $ is a primary ideal such that $ \rad(\A) = \p $ we say that $ \A $ is {$ \p $-primary}.

\begin{rems} \label{rem:power_primary} \quad 
	\begin{enumerate}
	\item The power of a prime ideal isn't always primary, for example if in $ R = \K[x,y,z] / (xy - z^2) $ we consider the prime ideal $ \p = (x,z) $ (it is prime since $ R / \p \simeq \K[y]$ that is an integral domain) we have that $ y $ is a zero divisor in $R/\p $ (since $ x $ is not zero and $ yx = z^2 = 0 $, since $z^2 \in \p^2 $) but it is not nilpotent since $ y^k \not \in \p^2 $ for all $ k>0 $
	\end{enumerate}
\end{rems}

We say tha an ideal $ \A \subseteq R $ has a \textbf{primary decomposition}  if there exists a finite set of primary ideal $ \{ \q_1 , ... , \q_n\} $ such that:
\begin{equation*}
	\A = \bigcap_{i=1}^n \q_i
\end{equation*}

In general such structure does not exists, but for $ R $ noetherian we can prove, using Noetherian induction and the concept of irreducible ideal, that every proper ideal has a primary decomposition.

\begin{deff}
	We say that a proper ideal $ \A $ is irreducible if it cannot be written as a proper intersection of ideal, i.e. :
	\begin{equation*}
		\A = \B \cap \Cc \Longrightarrow (\A = \B \text{ or } \A = \Cc)
	\end{equation*}
\end{deff}

\begin{lem}
	A proper ideal in a Noetherian ring $ R $ is always the intersection of a finite number of irreducible ideals.
\end{lem}

\begin{proof}
	Let $ \mathfrak{F} $ be the set of proper ideal such that the lemma is false. Let $ \A $ be a maximal ideal of $ \mathfrak{F} $, since it cannot be irreducible there exists $ \B $, $ \Cc $ strictly greater than $ \A $ ( so not in $ \mathfrak{F} $) such that $ \A = \B \cap \Cc $. This is absurd and so $ \mathfrak{F} $ is empty.
\end{proof}

\begin{lem}
	In a Noetherian ring every irreducible ideal is primary
\end{lem}

\begin{proof}
Modulo working in the quotient ring we can assume to work with the zero ideal. So we assume that the ideal $ 0 $ is irreducible and we consider $ x$, $y $ such that $ xy = 0 $ with $ y\neq 0 $, then $ x $ is a zerodivisor. So we have that $ y \in \Ann(x) $\footnote{For $ \Ann(x) $ we mean the annihilator of the principal ideal $ (x) $} and we consider the chain:
$$ \Ann(x)  \subseteq \Ann(x^2) \subseteq ...$$
And for the ascending chain condition there exists $ m $ with $ \Ann(x^m)= \Ann(x^{m+1})$. \\
Now consider $ a \in (x^m)\cap (y) $, then $ a = bx^m $ and $ a = cy $, so since $ y \in \Ann(x) $ we have $ 0 = cyx = ax = bx^m x=  bx^{m+1}$, so $ b \in  \Ann(x^{m+1}) = \Ann(x^m) $, then $ a = bx^n = 0 $. So $ (x^m)\cap (y)=0 $ and since $ 0 $ is irreducible and $ y\neq 0 $ then $ x^m=0 $. 
\end{proof}

Combining this two lemmas we have that the decomposition for Noetherian ring.

Now we need to achive some kind of uniqueness. First of all we say that a decomposition $ \A = \bigcap_{i=1}^n \q_i $ is \textbf{minimal} if:
\begin{enumerate}
\item $ \rad(\q_i) $ are all distinct
\item for all $ i $ we have $ \q_i \not \subseteq \bigcap_{j\neq i} \q_j $
\end{enumerate}

We can easly prove that from every decomposition we can obtain a minimal one using the following lemma:

\begin{lem}
If $ \A $ and $ \B $ are $ \p $-primary then $ \A \cap \B $ is $ \p $-primary
\end{lem}

Infact we can group the primaty ideal to get $ 1. $ and omit the superfluous terms to get $ 2. $

So we have two theorem of uniqueness for the prime \textit{associated}\footnote{not a random word} to a particular decomposition. 

\begin{teo}[First uniqueness theorem]
Let $ R $ be a Noetherian ring and $ \A $ an ideal with minimal decomposition $ \bigcap_{i=1}^n \q_i $, where $ \q_i $ is $ \p_i $-primary, then:
\[ \Ass(R/ \A) = \{ \p_1 , ... , \p_n \} \]
and so the set of primes $ \{ \p_1 , ... , \p_n \} $ is uniquely determined by the ideal

\end{teo}

\begin{teo}[Second uniqueness theorem]
Let $ R $ be a ring and $ \A $ an ideal with minimal decomposition $ \bigcap_{i=1}^n \q_i $, where $ \q_i $ is $ \p_i $-primary, then if $ p_i $ is a minimal element of $ \{ \p_1 , ... , \p_n \} $ $ \q_i $ is uniquely determined by the ideals $ \A $ and $ \p_i $.
%TODO Non so se lasciare sta roba tanto non serve
In particular if $ \phi : R \to R_{\p_i} = S^{-1} R$ is the canonical injection (where $ S = R \setminus \p_i $) we have
\[ 
\q_i = \phi ^{-1} ( S^{-1} \A  )
\]

\end{teo}


\subsection{Sybolic power}

\begin{deff}
	Let $ R $ be a noetherian ring and $ I $ an ideal. Given an integer $ m $ we define the \textbf{$ m $-th symbolic power} of $ I $ as:
	\begin{equation}\label{key}
		I^{(m)} = \bigcap_{\p \in \Ass(R/I) } (I^m R_\p \cap R)
	\end{equation}
\end{deff}

\subsection{Zarisky-Nagata Theorem}
Why do we studty symbolic power? The Zarisky-Nagata Theorem give a geometric interpretation of its significance.

\subsection{$ I^r \subseteq I^{(m)}$ }

\section{The Containment Problem}

\subsection{The Waldschmidt constant}

\newpage
\printbibliography

\end{document}
