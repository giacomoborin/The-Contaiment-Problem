\documentclass[]{book}
\usepackage[T1]{fontenc}
\usepackage[utf8]{inputenc}
\usepackage[english]{babel}
\usepackage{amssymb,amsmath,amsthm,mathtools} 
\usepackage{tikz-cd,wrapfig}
\usepackage{tcolorbox}

% Theorem definitons 
\theoremstyle{plain}
\newtheorem{teo}{Theorem}[section]
\newtheorem{lem}[teo]{Lemma}
\newtheorem{prop}[teo]{Proposition}
\newtheorem{cor}[teo]{Corollary}
\newtheorem*{form}{Formula}

\theoremstyle{remark}
\newtheorem{rem}{Remark}
\newtheorem{rems}[rem]{Remarks}

\theoremstyle{definition}
\newtheorem{deff}[teo]{Definiton}
\newtheorem{idea}{Idea}
\newtheorem*{nota}{Notation}


%Bibliography
\usepackage[style=numeric, maxnames=4,backend=bibtex]{biblatex}
% other styles: numeric authortitle
\addbibresource{biblio.bib}
\usepackage{hyperref}


% Commands 
\newcommand{\Z}{\mathbb{Z}}
\newcommand{\F}{\mathbb{F}}
\newcommand{\K}{\mathbb{K}}
\newcommand{\ZZ}[1]{\mathbb{Z}_{#1}}
\newcommand{\Q}{\mathbb{Q}}
\newcommand{\C}{\mathbb{C}}
\newcommand{\R}{\mathbb{R}}
\newcommand{\PP}{\mathbb{P}}

\newcommand{\p}{\mathfrak{p}}
\newcommand{\q}{\mathfrak{q}}
\newcommand{\mm}{\mathfrak{m}}
\newcommand{\A}{\mathfrak{a}}
\newcommand{\B}{\mathfrak{b}}
\newcommand{\Cc}{\mathfrak{c}}


\DeclareMathOperator*{\eqb }{=}
\DeclareMathOperator{\ord}{ord}
\DeclareMathOperator{\rad}{rad}
\DeclareMathOperator{\Ann}{Ann}
\DeclareMathOperator{\Ass}{Ass}
\DeclareMathOperator{\Spec}{Spec}


%Environment
\newenvironment{claim}[1]{\par\noindent\underline{Claim:}\space#1}{}
\newenvironment{claimproof}[1]{\par\noindent\underline{Proof:}\space#1}{\hfill $\blacksquare$}



%opening
\title{The Containment Problem, 
\\a general introduction and 
\\the particular case of Steiner configurations ideals}
\author{Giacomo Borin}



\begin{document}

\frontmatter
\maketitle
\tableofcontents


\mainmatter
\chapter{Inroduction and Symbolic Powers}
%TODO Should I say that R is always commutative with unity


\section{Associated primes}

Let $ R $ be a commutative ring with unity, and $ \A $,$ \B $ two ideal, we say that the ideal
\begin{equation*}
	(\A : \B) = \{ x \in R \,|\, x\B \subseteq \A  \}
\end{equation*}
 \nocite{AMCD}
is the \textit{ideal quotient}. For the case in which $ \A $  is the null ideal $ 0 $ we define the \textbf{annihilator} of $ \B $ as:
\begin{equation*}
	\Ann_R(\B) = (0 : \B) = \{ x \in R \,|\, x\B = 0  \}
\end{equation*}
We can obviously omit the index $ R $ if it is clear by the context. In general given an $ R  $-module $ M $ and a set $ S \subseteq M $ non empty we can define its annihilator as:
\begin{equation*}
	\Ann_R (S) = \{ x \in R \,| xS = 0  \} = \{ x \in R \,|\, \forall s \in S \; xs = 0  \}
\end{equation*}

\begin{deff}[Associated Prime]
Let $ M $ be an $ R $-module. A prime ideal $ \p \subseteq R $ is an \textbf{associated prime} of $ M $ if there exists a non-zero element $ a \in M $ such that $ \p = \Ann_R (a)$. \\
We define $ \Ass_R(M) $ as the set of the associated primes of $ M $.\\
For an ideal $ I $ we say that a prime is associated to $ I $ if it is associated to the $ R $-module $ R/I $.
\end{deff}

\begin{rem}
	Another name for associated ideal used by the Bourbaki group is \textit{assasin}, a word play between associated and annihilator. %todo check this
\end{rem}



\section{Primary decomposition}

%TODO Mettici qualcosa di meglio dai
We would like to have some sort of factorization for the ideals of a ring, more general than the \textit{unique factorization domains}, in fact this is useful only for principal ideals. With this objective \textbf{primary decomposition} was introduced. 

Now I will recall some of the principal result on this topics, contained in \cite[Section 7]{Reid} and \cite[Section 4 and Page 83]{AMCD}

\begin{deff}
	An ideal $ \A $ in a ring $ R $ is said primary if $ R/\A$ is different from zero and all its zerodivisors are nilpotent, otherwhise we can express this as:
	\begin{equation*}
		fg \in \aa \Longrightarrow f \in \A \text{ or } g^n \in \A \text{ for some } n >0
	\end{equation*}
\end{deff}

It is obvious that the radical of a primary ideal is a prime ideal, infact given $ fg \in \rad(\A)  $ we have $ (fg)^m = f^m g^m \in \A $ for $ m>0 $, and so $ f^m \in \A \Rightarrow f \in \rad(\A)$ or exists $ n>0 $ such that $ g^{mn} \in \A \Rightarrow g \in \rad(\A) $.

If $ \A $ is a primary ideal such that $ \rad(\A) = \p $ we say that $ \A $ is {$ \p $-primary}.

\begin{rems} \label{rem:power_primary} \quad 
	\begin{enumerate}
	\item The power of a prime ideal isn't always primary, for example if in $ R = \K[x,y,z] / (xy - z^2) $ we consider the prime ideal $ \p = (x,z) $ (it is prime since $ R / \p \simeq \K[y]$ that is an integral domain) we have that $ y $ is a zero divisor in $R/\p $ (since $ x $ is not zero and $ yx = z^2 = 0 $, since $z^2 \in \p^2 $) but it is not nilpotent since $ y^k \not \in \p^2 $ for all $ k>0 $
	\end{enumerate}
\end{rems}

We say tha an ideal $ \A \subseteq R $ has a \textbf{primary decomposition}  if there exists a finite set of primary ideal $ \{ \q_1 , ... , \q_n\} $ such that:
\begin{equation*}
	\A = \bigcap_{i=1}^n \q_i
\end{equation*}

In general such structure does not exists, but for $ R $ noetherian we can prove, using Noetherian induction and the concept of irreducible ideal, that every proper ideal has a primary decomposition.

\begin{deff}
	We say that a proper ideal $ \A $ is irreducible if it cannot be written as a proper intersection of ideal, i.e. :
	\begin{equation*}
		\A = \B \cap \Cc \Longrightarrow (\A = \B \text{ or } \A = \Cc)
	\end{equation*}
\end{deff}

\begin{lem}
	A proper ideal in a Noetherian ring $ R $ is always the intersection of a finite number of irreducible ideals.
\end{lem}

\begin{proof}
	Let $ \mathfrak{F} $ be the set of proper ideal such that the lemma is false. Let $ \A $ be a maximal ideal of $ \mathfrak{F} $, since it cannot be irreducible there exists $ \B $, $ \Cc $ strictly greater than $ \A $ ( so not in $ \mathfrak{F} $) such that $ \A = \B \cap \Cc $. This is absurd and so $ \mathfrak{F} $ is empty.
\end{proof}

\begin{lem}
	In a Noetherian ring every irreducible ideal is primary
\end{lem}

\begin{proof}
Modulo working in the quotient ring we can assume to work with the zero ideal. So we assume that the ideal $ 0 $ is irreducible and we consider $ x$, $y $ such that $ xy = 0 $ with $ y\neq 0 $, then $ x $ is a zerodivisor. So we have that $ y \in \Ann(x) $\footnote{For $ \Ann(x) $ we mean the annihilator of the principal ideal $ (x) $} and we consider the chain:
$$ \Ann(x)  \subseteq \Ann(x^2) \subseteq ...$$
And for the ascending chain condition there exists $ m $ with $ \Ann(x^m)= \Ann(x^{m+1})$. \\
Now consider $ a \in (x^m)\cap (y) $, then $ a = bx^m $ and $ a = cy $, so since $ y \in \Ann(x) $ we have $ 0 = cyx = ax = bx^m x=  bx^{m+1}$, so $ b \in  \Ann(x^{m+1}) = \Ann(x^m) $, then $ a = bx^n = 0 $. So $ (x^m)\cap (y)=0 $ and since $ 0 $ is irreducible and $ y\neq 0 $ then $ x^m=0 $. 
\end{proof}

Combining this two lemmas we have that the decomposition for Noetherian ring. In literature we say that a commutative ring is a \textbf{Lasker Ring} if every ideal has a primary decomposition, so we can state that:

\begin{teo}[Lasker-Noether]
A Noetherian Ring is also a Lasker Ring
\end{teo}
Now we need to achive some kind of uniqueness. First of all we say that a decomposition $ \A = \bigcap_{i=1}^n \q_i $ is \textbf{minimal} if:
\begin{enumerate}
\item $ \rad(\q_i) $ are all distinct
\item for all $ i $ we have $ \q_i \not \subseteq \bigcap_{j\neq i} \q_j $
\end{enumerate}

We can easly prove that from every decomposition we can obtain a minimal one using the following lemma:

\begin{lem}
If $ \A $ and $ \B $ are $ \p $-primary then $ \A \cap \B $ is $ \p $-primary
\end{lem}

Infact we can group the primaty ideal to get $ 1. $ and omit the superfluous terms to get $ 2. $

So we have two theorem of uniqueness for the prime \textit{associated}\footnote{not a random word} to a particular decomposition. 

\begin{teo}[First uniqueness theorem]
Let $ R $ be a Noetherian ring and $ \A $ an ideal with minimal decomposition $ \bigcap_{i=1}^n \q_i $, where $ \q_i $ is $ \p_i $-primary, then:
\[ \Ass(R/ \A) = \{ \p_1 , ... , \p_n \} \]
and so the set of primes $ \{ \p_1 , ... , \p_n \} $ is uniquely determined by the ideal

\end{teo}

This theorem show the strong ralation that we have between the associated prime ideal and the primary decomposition for Noetherian ring

\begin{teo}[Second uniqueness theorem]
Let $ R $ be a ring and $ \A $ an ideal with minimal decomposition $ \bigcap_{i=1}^n \q_i $, where $ \q_i $ is $ \p_i $-primary, then if $ p_i $ is a minimal element of $ \{ \p_1 , ... , \p_n \} $ $ \q_i $ is uniquely determined by the ideals $ \A $ and $ \p_i $.
%TODO Non so se lasciare sta roba tanto non serve
In particular if $ \phi : R \to R_{\p_i} = S^{-1} R$ is the canonical injection (where $ S = R \setminus \p_i $) we have
\[ 
\q_i = \phi ^{-1} ( S^{-1} \A  )
\]

\end{teo}


\section{Sybolic power}

	Lets consider an homogeneous polynomial ring $ k[x_0 , ... , x_n] $, it is easy to se that if we consider a variety $ X $ with it's coordinate ring $ R = k[X] $ and a point $ p \in X $ (associated to the maximal ideal $ \mm_p $)  we have that:
	
	\begin{equation}\label{eq:max_pow}
		\mm_p^n = \{ f \in k[X] \text{ such that } f \text{ vanishes in } p \text{ with multiplicity } n\}
	\end{equation}
	
	For general ideal we don't have similiar properties for the normal power of ideal, so we define the \textbf{symbolic power}. %TODO Qualcosa di più qui
	First of all given a prime ideal $ \p $ in a Noetherian Ring $ R $ we can define the $ n $-th symbolic power of $ \p $ as:
	\begin{equation}\label{eq:sym_pow_p1}
		\p^{(n)} = \{ r \in R \text{ such that exists } s \in R \setminus \p \text{ with } sr \in \p^n \}
	\end{equation}
	This definition show clearly the idea between the symbolic power, but is not easy to work with. We can have another equivalent definition that use the localization on the prime ideal $ R_\p $. Infact we can see it as the contraction of $ \p^n R_\p $ over $ R $:
	\begin{equation}\label{eq:sym_pow_p2}
		\p^{(n)} = \p^n R_\p \cap R
	\end{equation}
	In general the generic and symbolic power are different concept. It is obvious that $ \p^{n} \subset \p^{(n)} $ since $ 1 \not \in \p $. For the other direction we can costruct a counter example with the following proposition:
	\begin{prop}\label{prop:sym_is_primary}
		$ \p^{(n)} $ is the smallest $ \p $-primary ideal that contain $ \p^n $
	\end{prop}
	\begin{proof}\quad \\
		\textit{Primary}: If $ xy \in  \p^{(n)}$ with $ x \not \in  \p^{(n)}$ we have that exists $ s \not \in \p $ with $ sxy \in \p^n $. Suppose that $ sy \not \in \p $, so $ (sy)x \in \p^n $ and then $ x \in \p^{(n)} $ that is absurd, so $ sy \in \p \Rightarrow (sy)^n \in \p^n \Rightarrow s^n y^n \in \p^n$. Since $ \p $ is prime $ s^n \not \in \p $ and so $ y^n \in \p^{(n)} $. \\
		\textit{$p$-primary}: Infact $ \p^{(n)} \subset \p $ and so $  \rad  (\p^{(n)} )\subset \rad (\p) = \p$. Also if $ x \in \p $ we have $ x^n \in \p^n \subset \p^{(n)}  $ and so $ x \in \p^{(n)}  $.\\
		\textit{Minimal}: If $ \q $ is $ \p $-primary and contains $ \p^n $, then for $ r \in \p^{(n)}  $ there exists $ s \not \in \p \supset \q $ with $ sr \in \p^{(n)}  \subset \q $, and so since $ s \not \in \q  $ exists $ k $ such that $ r^k \in \q $. If $ k=1 $ we have finished otherwhise we terminate by induction using $ r r^{k-1} \in \q $.
	\end{proof}
	Using the same example from Remark \ref{rem:power_primary} we can observe that necessarly $ p^2 \neq \p^{(2)}  $ since the first one isn't prymary. 
	
	\begin{rems}
		\begin{itemize}
		\item The proposition \ref{prop:sym_is_primary} establish a new equivalent definition for the symbolic power, more in line to the use of this ideal in the Zarisky-Nakata Theorem.
		\item Using the properties of localization, like \cite[Proposition 4.8]{AMCD} and working with the contraction we would have speed up the proof.
		%TODO Motiva perchè non lo hai fatto/di che lo farai dopo
		\end{itemize}
	\end{rems}
	
Now we can see the actual definition of this concept for a general ideal.

\begin{deff}
	Let $ R $ be a noetherian ring and $ I $ an ideal. Given an integer $ m $ we define the \textbf{$ m $-th symbolic power} of $ I $ as:
	\begin{equation}\label{eq:sym_pow_def}
		I^{(m)} = \bigcap_{\p \in \Ass(R/I) } (I^m R_\p \cap R)
	\end{equation}
\end{deff}



\section{Zarisky-Nagata Theorem}
Why do we study symbolic power? The Zarisky-Nagata Theorem give a geometric interpretation of its significance.

\begin{teo}[Zarisky-Nagata Theorem \cite{Zar49, Nagata62}]
	If $ R = k[x_0 , ... , x_n] $ is a polynomial ring and $ \p $ is a prime ideal then:
	\begin{equation}\label{eq:zar_nag_teo}
	\p^{(m)} = \bigcap_{\substack{ \mm \in \mm\Spec (R)\\ \p \subset \mm}} \mm ^n
	\end{equation}
\end{teo}

Using the equation \ref{eq:max_pow} we can see that in this case the $ n $-th symbolic power of a prime ideal represents the ideal composed by all the polynomials vanishing on the variety with a multiplicity of $ n $, also indicated with the notation:
\begin{equation}\label{eq:ideal_vanish}
	I^{\left<n\right>} = \{ f \in R \text{ that vanishes on } \mathcal{V}(I) \text{ with multiplicity } n\}
\end{equation}

%TODO Posso prendere roba dagli articoli


\section{Fat Points}

Let's consider now an object of interest, that has particular relations with the symbolic powers.\\
Let $ k $ be a field and $ \PP^N$ the $ N $-th projective space over $ k $, consider now the distincts points $ p_1, ... ,p_k \in  \PP^N$ and some positive integers $ m_1 , ... ,m_k $. If we consider the defining ideals $ I(p_1), ... , I(p_k)  \subset k [\PP^N]$, representing the homogeneous polynomials vanishing on the point (before we have also used the notation $ \mm_{p_i}$ that emphasise their role as maximal ideals) we can define another ideal:
\begin{equation}\label{eq:fat_pt}
	I = \bigcap_{i=1}^k I(p_i)^m_i \subset k [\PP^N]
\end{equation}
Since is intersection of homogenous ideals $ I $ is also homogeneous and we can use it to define a $ 0 $ dimensional subscheme $ Z \subset \PP^N $, called \textbf{fat point subscheme} and indicated as 
\[
Z = m_1 p_1 + ... +  m_k p_k
\]
And we will denote $ I $ as $ I(Z) $. This ideal represents the homogeneous polynomials that vanishes on $ p_i $ with multiplicity $ m_i $ for all $ i = 1, ... , k $. The support of the scheme is the set of points $ \{ p_1 , ... , p_k\} $.\\
The simbolic power of $ I(Z) $ has the particular from:
\[
I(Z)^{(m)} = \bigcap_{i=1}^k I(p_i)^{m_im}
\]
and so clearly represents the functions vanishing on $ p_i $ with multiplicity $ mm_i $ for all $ i = 1, ... , k $. 

%TODO Scrivere dove li useremo

\begin{tcolorbox}
Potrebbe servirmi l'articolo \cite{Gimigliano89} ma per ora non lo trovo (forse).
Esiste un modo per provare che è 0 dimensionale? %TODO Sono sicuro di sapere cosa vuol dire?
\end{tcolorbox}


\section{Computation of Symbolic powers}

%TODO Actual computation of symbolic powers

\section{Relation between normal and symbolic power}
\begin{tcolorbox}

\textbf{Cosa trovata nell'intrdouduzione di \cite{BocciHrabourne10} , sarebbe carino riprendere un discorso simile}

Consider a homogeneous ideal I in a polynomial ring k[PN ]. Taking powers of I is a natural algebraic construction, but it can be difficult to understand their structure geometrically (for example, knowing generators of Ir does not make it easy to know its primary decomposition). On the other hand, symbolic powers of I are more natural geometrically than algebraically. For example, if I is a radical ideal defining a finite set of points p 1, . . . , ps  PN , then its mth symbolic power I(m) is generated by all forms vanishing to order at least m at each point pi, but it is not easy to write down specific generators for I(m), even if one has generators for I.

A natural question that arises is the relation between the two powers.
\end{tcolorbox}

\begin{teo}
	If $ R $ is a Noetherian reduced ring then $ I^r \subseteq I^{(m)}$  if and only if $ r \geq m $
\end{teo}
\begin{proof}
	Since if $ I=0 $ it is obvious we assume it to be non zero.\\
	The first implication is easy, infact for all $ \p $ we have that $ I^m \subset I^mR_\p \cap R $ since $ I \subset \p $\footnote{So nothing become zero} and $ 1 \not \in \p $.\\
	Suppose now that $ I^r \subseteq I^{(m)}$ and $ r < m $. Consider an associated prime $ \p $, then if we consider the localization $ I_\p $ we have:
	\begin{itemize}
	\item $ (I_\p)^r = (I^r)_\p $ it is obvious
	%TODO Sono la stessa notazione I_\p e IR_\p ??????
	\item $ (I_\p)^m \supset (I^{(m)})_\p $, because $ I^{(m)} \subset I^mR_\p \cap R = I^m_\p \cap R $ and thus passing to the localization (again for the second term) we get the containment.
	\end{itemize}
	So composing this with the containment hypotesis (localized) we get $ (I_\p)^r = (I^r)_\p \subset (I^{(m)})_\p \subset (I_\p)^m $. Since the other inclusion is obvious we have $ I_\p^m= I_\p^r $, considering the intermediary power we have that $ I_\p I_\p^r =I_\p^r $, thus we can use the Nakayama Lemma:
	
	\begin{lem}[Nakayama]
		Given an ideal $ \A $ of a commutative ring with unity $ A $ and $ M $ a finetely-generated module over $ A $ with $ IM = M $, then there exists a $ x \in A $ such that $ a \equiv 1 \mod I $ and $ rM = 0 $
	\end{lem}
	%TODO Scolta non è che l'estensione di I in R_\p e l'estensione non sono la stessa cosa?
	In our case the ring is $ R_\p $, the ideal is $ I_\p $ and the module is $ I_\p^{r} $. So since $ R_p $ is a local ring with maximal ideal the localization of $ \p $, that is also the Jacobson Ideal, and $ I_\p ^r $ is contained in it. So for the characterization of the Jacobson ideal the element $ x $ is invertible and then $ I_\p^r = 0 $, but since we inherit that $ R_\p $ is a reduced ring this is possible only if $ I_\p = 0$. Since this is true for all the prime ideal (for the non associated one it is obvious since they intersect $ R \setminus I $) and this is a local property (\cite[Proposition 3.8]{AMCD}) we have $ I = 0 $, absurd.
\end{proof} 
Sadly the other direction of the containment isn't that easy, and it's an open question and in the last years was studied for some classes of ideals.

\chapter{The Containment Problem}

\section{The Waldschmidt constant}

\chapter{Some conjectures for the Containment Problem}

\chapter{Steiner Configuration ideal}
Interesting ideals, am example is the Fano Plane

\chapter{Coloration of Steiner Configuration ideal}
Fun fuct, I can use the cp to check if there exists a colorarion


\backmatter
\printbibliography
\end{document}
