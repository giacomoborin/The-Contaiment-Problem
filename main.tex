\documentclass[notitlepage, a4]{book}
\usepackage[T1]{fontenc}
\usepackage[utf8]{inputenc}
\usepackage[english]{babel}
\usepackage{amssymb,amsmath,amsthm,mathtools} 
\usepackage{tikz-cd,wrapfig}
\usepackage{tcolorbox}
\tcbuselibrary{skins}
\usepackage[Bjarne]{fncychap}
\usepackage{graphicx}
%\graphicspath{ {images/} }
\usepackage{listings}

\usepackage{fancyhdr}
\usepackage{multirow}
\usepackage{xcolor}
\usepackage[all,2cell]{xy}
\usepackage{float}
\usepackage{csquotes}

\usepackage{tikz}
\usetikzlibrary{cd}


% Theorem definitons 
\theoremstyle{plain}
\newtheorem{teo}{Theorem}[section]
\newtheorem{lem}[teo]{Lemma}
\newtheorem{prop}[teo]{Proposition}
\newtheorem{cor}[teo]{Corollary}
\newtheorem*{form}{Formula}
\newtheorem{conj}[teo]{Conjecture}

\theoremstyle{remark}
\newtheorem{rem}{Remark}
\newtheorem{rems}[rem]{Remarks}
\newtheorem{que}[rem]{Question}
\newtheorem{ex}[rem]{Example}
\newtheorem{exs}[rem]{Examples}

\theoremstyle{definition}
\newtheorem{deff}[teo]{Definiton}
\newtheorem{idea}{Idea}
\newtheorem*{nota}{Notation}


%Bibliography
\usepackage[style=alphabetic, maxnames = 8, backend=bibtex,giveninits=true,isbn=false,eprint=false]{biblatex}
% other styles: numeric authortitle
\addbibresource{biblio.bib}
\usepackage[hidelinks]{hyperref}
%\hypersetup{frenchlinks=true}
\hypersetup{
  colorlinks   = true, %Colours links instead of ugly boxes
  urlcolor     = gray, %Colour for external hyperlinks
  linkcolor    = blue, %Colour of internal links
  citecolor   = green %Colour of citations
}


% Commands 

\newcommand{\parder}[2]{ \frac{\partial #1}{\partial #2} }
\newcommand{\Z}{\mathbb{Z}}
\newcommand{\F}{\mathbb{F}}
\newcommand{\K}{\mathbb{K}}
\newcommand{\ZZ}[1]{\mathbb{Z}_{#1}}
\newcommand{\Q}{\mathbb{Q}}
\newcommand{\C}{\mathbb{C}}
\newcommand{\R}{\mathbb{R}}
\newcommand{\PP}{\mathbb{P}}
\newcommand{\HH}{\mathcal{H}}
\newcommand{\MM}{\mathcal{M}}

\newcommand{\p}{\mathfrak{p}}
\newcommand{\q}{\mathfrak{q}}
\newcommand{\mm}{\mathfrak{m}}
\newcommand{\A}{\mathfrak{a}}
\newcommand{\B}{\mathfrak{b}}
\newcommand{\Cc}{\mathfrak{c}}

\newcommand{\cont}[2]{ I^{(#1)} \subseteq I^{#2}}
\newcommand{\mcont}[3]{ I^{(#1)} \subseteq \MM^{#2} I^{#3}}


\DeclareMathOperator*{\eqb }{=}
\DeclareMathOperator{\ord}{ord}
\DeclareMathOperator{\rad}{rad}
\DeclareMathOperator{\Ann}{Ann}
\DeclareMathOperator{\Ass}{Ass}
\DeclareMathOperator{\Spec}{Spec}
\DeclareMathOperator{\hgt}{ht}
\DeclareMathOperator{\co}{co}
\DeclareMathOperator{\reg}{reg}


%Environment
\newenvironment{claim}[1]{\par\noindent\underline{Claim:}\space#1}{}
\newenvironment{claimproof}[1]{\par\noindent\underline{Proof:}\space#1}{\hfill $\blacksquare$}

%tcolorbox
\newcounter{que}
\newtcolorbox{tboxprop}[1][]
{
title=\stepcounter{que}\theque : Possibile aggiunta,
colback=red!20!white, 
colframe=red!50!white
}
\newtcolorbox{tboxque}[1][]
{
title=\stepcounter{que}\theque : Domanda,
colback=blue!20!white, 
colframe=blue!50!white
}

\newtcolorbox{tboxtodo}[1][]
{
title=\stepcounter{que}\theque : Cosa da fare,
coltitle=white,
colback=yellow!20!white, 
colframe=orange!50!white
}



%opening
\title{The Containment Problem, 
\\a general introduction and 
\\the particular case for Steiner systems}
\author{Giacomo Borin}



\begin{document}
\frontmatter

\pagestyle{fancy}
\fancyhf{}

\renewcommand{\headrulewidth}{1.5pt}
\renewcommand{\footrulewidth}{1.5pt}
\cfoot{{\fontfamily{phv} \small Academic Year 2020-21}}

\include{frontespizio}

\renewcommand{\headrulewidth}{1pt}
\renewcommand{\footrulewidth}{0pt}
\renewcommand{\chaptermark}[1]{\markboth{#1}{}}
\rhead{\leftmark}
\rfoot{\thepage}

\tableofcontents

%\chapter*{Introduction}
%\label{sec:intro}
%\addcontentsline{toc}{chapter}{Introduction}
%
%\chaptermark{Introduction}
%
%The Containment 
%In this thesis 

\mainmatter



\chapter{Preliminaries and Symbolic Powers}
%TODO Should I say that R is always commutative with unity


\section{Associated primes}

Let $ R $ be a commutative ring with unity, and $ \A $,$ \B $ two ideal, we say that the ideal
\begin{equation*}
	(\A : \B) = \{ x \in R \,|\, x\B \subseteq \A  \}
\end{equation*}
 \nocite{AMCD}
is the \textit{ideal quotient}. For the case in which $ \A $  is the null ideal $ 0 $ we define the \textbf{annihilator} of $ \B $ as:
\begin{equation*}
	\Ann_R(\B) = (0 : \B) = \{ x \in R \,|\, x\B = 0  \}
\end{equation*}
We can obviously omit the index $ R $ if it is clear by the context. In general given an $ R  $-module $ M $ and a set $ S \subseteq M $ non empty we can define its annihilator as:
\begin{equation*}
	\Ann_R (S) = \{ x \in R \,| xS = 0  \} = \{ x \in R \,|\, \forall s \in S \; xs = 0  \}
\end{equation*}

\begin{deff}[Associated Prime]
Let $ M $ be an $ R $-module. A prime ideal $ \p \subseteq R $ is an \textbf{associated prime} of $ M $ if there exists a non-zero element $ a \in M $ such that $ \p = \Ann_R (a)$. \\
We define $ \Ass_R(M) $ as the set of the associated primes of $ M $.\\
For an ideal $ I $ we say that a prime is associated to $ I $ if it is associated to the $ R $-module $ R/I $.
\end{deff}

Between the associated primes of an ideal we distinguish the minimal elements, that are called \textbf{isolated primes}, whilst the other one are said \textbf{embedded primes}.

We can define also the minimal primes of the ideal $ I $, that are the minimal ones that contains $ I $. In Noetherian rings these concepts are redundant, in fact with the following proposition we have that minimal and isolated are equivalent. 

\begin{prop} \label{prop:minprimes}
	For a Noetherian ring $ R$, the minimal primes of $ R $ are among the associated primes of $ R $
\end{prop}

%\begin{tboxque}
%	Non so se inserire una dimostrazione di questo, ad oggi non ne ho trovata una che usi solo cose introdotte qui, ad eccezione di \href{https://mathoverflow.net/questions/30792/a-direct-proof-that-minimal-primes-are-associated}{questa}, che usa la decomposizione primaria. 
%\end{tboxque}
%
%	1) domanda: non inserirla. ma la nozione di associated prime viene proprio dalla decomposizione primaria, quindi sarebbe una proof ultranaturale, ma evitiamola per motivo di lunghezza
%

\begin{rem}
	Another name for associated ideal used by the Bourbaki group is \textit{assassin} or \textit{assassinator}, a word play between associated and annihilator. %todo check this
\end{rem}

\section{Primary decomposition}

%TODO Mettici qualcosa di meglio dai
We would like to have some sort of factorization for the ideals of a ring, more general than the \textit{unique factorization domains}, in fact this is useful only for principal ideals. With this objective \textbf{primary decomposition} was introduced. 

Now I will recall some of the principal result on this topics, contained in \cite[Section 7]{Reid} and \cite[Section 4 and Page 83]{AMCD}

\begin{deff}
	An ideal $ \A $ in a ring $ R $ is said primary if $ R/\A$ is different from zero and all its zero divisors are nilpotent, otherwise we can express this as:
	\begin{equation*}
		fg \in \aa \Longrightarrow f \in \A \text{ or } g^n \in \A \text{ for some } n >0
	\end{equation*}
\end{deff}

It is obvious that the radical of a primary ideal is a prime ideal, in fact given $ fg \in \rad(\A)  $ we have $ (fg)^m = f^m g^m \in \A $ for $ m>0 $, and so $ f^m \in \A \Rightarrow f \in \rad(\A)$ or exists $ n>0 $ such that $ g^{mn} \in \A \Rightarrow g \in \rad(\A) $.

If $ \A $ is a primary ideal such that $ \rad(\A) = \p $ we say that $ \A $ is {$ \p $-primary}.

\begin{rems} \label{rem:power_primary} \quad 
	\begin{enumerate}
	\item The power of a prime ideal isn't always primary, for example if in $ R = \K[x,y,z] / (xy - z^2) $ we consider the prime ideal $ \p = (x,z) $ (it is prime since $ R / \p \simeq \K[y]$ that is an integral domain) we have that $ y $ is a zero divisor in $R/\p $ (since $ x $ is not zero and $ yx = z^2 = 0 $, since $z^2 \in \p^2 $) but it is not nilpotent since $ y^k \not \in \p^2 $ for all $ k>0 $
	\end{enumerate}
\end{rems}

We say that an ideal $ \A \subseteq R $ has a \textbf{primary decomposition}  if there exists a finite set of primary ideal $ \{ \q_1 , ... , \q_n\} $ such that:
\begin{equation*}
	\A = \bigcap_{i=1}^n \q_i
\end{equation*}

In general such structure does not exists, but for $ R $ noetherian we can prove, using Noetherian induction and the concept of irreducible ideal, that every proper ideal has a primary decomposition.

\begin{deff}
	We say that a proper ideal $ \A $ is irreducible if it cannot be written as a proper intersection of ideal, i.e. :
	\begin{equation*}
		\A = \B \cap \Cc \Longrightarrow (\A = \B \text{ or } \A = \Cc)
	\end{equation*}
\end{deff}

\begin{lem}
	A proper ideal in a Noetherian ring $ R $ is always the intersection of a finite number of irreducible ideals.
\end{lem}

\begin{proof}
	Let $ \mathfrak{F} $ be the set of proper ideal such that the lemma is false. Let $ \A $ be a maximal ideal of $ \mathfrak{F} $, since it cannot be irreducible there exists $ \B $, $ \Cc $ strictly greater than $ \A $ ( so not in $ \mathfrak{F} $) such that $ \A = \B \cap \Cc $. This is absurd and so $ \mathfrak{F} $ is empty.
\end{proof}

\begin{lem}
	In a Noetherian ring every irreducible ideal is primary
\end{lem}

\begin{proof}
Modulo working in the quotient ring we can assume to work with the zero ideal. So we assume that the ideal $ 0 $ is irreducible and we consider $ x$, $y $ such that $ xy = 0 $ with $ y\neq 0 $, then $ x $ is a zero divisor. So we have that $ y \in \Ann(x) $\footnote{For $ \Ann(x) $ we mean the annihilator of the principal ideal $ (x) $} and we consider the chain:
$$ \Ann(x)  \subseteq \Ann(x^2) \subseteq ...$$
And for the ascending chain condition there exists $ m $ with $ \Ann(x^m)= \Ann(x^{m+1})$. \\
Now consider $ a \in (x^m)\cap (y) $, then $ a = bx^m $ and $ a = cy $, so since $ y \in \Ann(x) $ we have $ 0 = cyx = ax = bx^m x=  bx^{m+1}$, so $ b \in  \Ann(x^{m+1}) = \Ann(x^m) $, then $ a = bx^n = 0 $. So $ (x^m)\cap (y)=0 $ and since $ 0 $ is irreducible and $ y\neq 0 $ then $ x^m=0 $. 
\end{proof}

Combining this two lemmas we have that the decomposition for Noetherian ring. In literature we say that a commutative ring is a \textbf{Lasker Ring} if every ideal has a primary decomposition, so we can state that:

\begin{teo}[Lasker-Noether]
A Noetherian Ring is also a Lasker Ring
\end{teo}
Now we need to achieve some kind of uniqueness. First of all we say that a decomposition $ \A = \bigcap_{i=1}^n \q_i $ is \textbf{minimal} if:
\begin{enumerate}
\item $ \rad(\q_i) $ are all distinct
\item for all $ i $ we have $ \q_i \not \subseteq \bigcap_{j\neq i} \q_j $
\end{enumerate}

We can easily prove that from every decomposition we can obtain a minimal one using the following lemma:

\begin{lem}
If $ \A $ and $ \B $ are $ \p $-primary then $ \A \cap \B $ is $ \p $-primary
\end{lem}

In fact we can group the primary ideal to get $ 1. $ and omit the superfluous terms to get $ 2. $

So we have two theorem of uniqueness for the prime \textit{associated}\footnote{not a random word} to a particular decomposition. 

\begin{teo}[First uniqueness theorem]
Let $ R $ be a Noetherian ring and $ \A $ an ideal with minimal decomposition $ \bigcap_{i=1}^n \q_i $, where $ \q_i $ is $ \p_i $-primary, then:
\[ \Ass(R/ \A) = \{ \p_1 , ... , \p_n \} \]
and so the set of primes $ \{ \p_1 , ... , \p_n \} $ is uniquely determined by the ideal

\end{teo}

This theorem show the strong relation that we have between the associated prime ideal and the primary decomposition for Noetherian ring. Also, it is possible to show that the factors $ \q_i $ depends only on the ideal and the primes $ \p_i $, in particular:

\begin{teo}[Second uniqueness theorem]
Let $ R $ be a ring and $ \A $ an ideal with minimal decomposition $ \bigcap_{i=1}^n \q_i $, where $ \q_i $ is $ \p_i $-primary, then if $ \p_i $ is a minimal element of $ \{ \p_1 , ... , \p_n \} $ $ \q_i $ is uniquely determined by the ideals $ \A $ and $ \p_i $.\\
%TODO Non so se lasciare sta roba tanto non serve
In particular if $ \phi : R \to R_{\p_i} = S^{-1} R$ is the canonical injection (where $ S = R \setminus \p_i $) we have
\[ 
\q_i = \phi ^{-1} ( S^{-1} \A  )
\]

\end{teo}


\section{Symbolic power}

	Lets consider an homogeneous polynomial ring $ k[x_0 , ... , x_n] $, it is easy to see that if we consider a variety $ X $ with it's coordinate ring $ R = k[X] $ and a point $ p \in X $ (associated to the maximal ideal $ \mm_p $)  we have that:
	
	\begin{equation}\label{eq:max_pow}
		\mm_p^n = \{ f \in k[X] \text{ such that } f \text{ vanishes in } p \text{ with multiplicity } n\}
	\end{equation}
	
	Sadly for a general ideal to get a similar results we can not rely on the normal power, so now we introduce a sharper object, the \textbf{symbolic power}, that has nicer geometric properties and we will see that in some way can answer to our problem. %TODO Qualcosa di più qui
	
%	\begin{tboxtodo}
%	Vorrei trovare qualche informazione di più sulla storia delle potenze simboliche, che non penso fossero state introdotte con questa idea, ad esempio ho visto che è usata per dimostrare il \textit{Krull's height theorem}
%	\end{tboxtodo}
%	
%	2) Cosa non fare: lo scopo e'essere brevi, altrimenti si deve tagliare qualche cosa
%		
	First of all given a prime ideal $ \p $ in a Noetherian Ring $ R $ we can define the $ n $-th symbolic power of $ \p $ as:
	\begin{equation}\label{eq:sym_pow_p1}
		\p^{(n)} = \{ r \in R \text{ such that exists } s \in R \setminus \p \text{ with } sr \in \p^n \}
	\end{equation}
	This definition show clearly the idea between the symbolic power, but is not easy to work with. We can have another equivalent definition that use the localization on the prime ideal $ R_\p $. In fact we can see it as the contraction of $ \p^n R_\p $ over $ R $:
	\begin{equation}\label{eq:sym_pow_p2}
		\p^{(n)} = \p^n R_\p \cap R
	\end{equation}
	In general the generic and symbolic power are different concept. It is obvious that $ \p^{n} \subset \p^{(n)} $ since $ 1 \not \in \p $. For the other direction we can construct a counter example with the following proposition:
	\begin{prop}\label{prop:sym_is_primary}
		$ \p^{(n)} $ is the smallest $ \p $-primary ideal that contain $ \p^n $
	\end{prop}
	\begin{proof}\quad \\
		\textit{Primary}: If $ xy \in  \p^{(n)}$ with $ x \not \in  \p^{(n)}$ we have that exists $ s \not \in \p $ with $ sxy \in \p^n $. Suppose that $ sy \not \in \p $, so $ (sy)x \in \p^n $ and then $ x \in \p^{(n)} $ that is absurd, so $ sy \in \p \Rightarrow (sy)^n \in \p^n \Rightarrow s^n y^n \in \p^n$. Since $ \p $ is prime $ s^n \not \in \p $ and so $ y^n \in \p^{(n)} $. \\
		\textit{$p$-primary}: In fact $ \p^{(n)} \subset \p $ and so $  \rad  (\p^{(n)} )\subset \rad (\p) = \p$. Also if $ x \in \p $ we have $ x^n \in \p^n \subset \p^{(n)}  $ and so $ x \in \p^{(n)}  $.\\
		\textit{Minimal}: If $ \q $ is $ \p $-primary and contains $ \p^n $, then for $ r \in \p^{(n)}  $ there exists $ s \not \in \p \supset \q $ with $ sr \in \p^{(n)}  \subset \q $, and so since $ s \not \in \q  $ exists $ k $ such that $ r^k \in \q $. If $ k=1 $ we have finished otherwise we terminate by induction using $ r r^{k-1} \in \q $.
	\end{proof}
	Using the same example from Remark \ref{rem:power_primary} we can observe that necessary $ p^2 \neq \p^{(2)}  $ since the first one isn't primary. 
	
	\begin{rems}
		\begin{itemize}
		\item The proposition \ref{prop:sym_is_primary} establish a new equivalent definition for the symbolic power, more in line to the use of this ideal in the Zariski-Nagata Theorem.
		\item Using the properties of localization, like \cite[Proposition 4.8]{AMCD} and working with the contraction we would have speed up the proof.
		%TODO Motiva perchè non lo hai fatto/di che lo farai dopo
		\end{itemize}
	\end{rems}
	
Now we can see the actual definition of this concept for a general ideal.

\begin{deff}
	Let $ R $ be a noetherian ring and $ I $ an ideal. Given an integer $ m $ we define the \textbf{$ m $-th symbolic power} of $ I $ as:
	\begin{equation}\label{eq:sym_pow_def}
		I^{(m)} = \bigcap_{\p \in \Ass(R/I) } (I^m R_\p \cap R)
	\end{equation}
\end{deff}

\begin{rems} \label{rem:symb_basic}
Working on the localization over the associate ideals $ \Ass(R/I) $ is possible to show some simple properties for the symbolic power:
\begin{enumerate}
\item $ I^{(1)}=I $
\item $ I^{(a)} \subseteq I^{(b)} $ for all $ a > b $
\item $ I^{(a)}I^{(b)}\subseteq I^{(a+b)} $ for all $ a,b $ positive integers
\end{enumerate}
\end{rems}



\section{Zariski-Nagata Theorem}
Why do we study symbolic power? The Zariski-Nagata Theorem give a geometric interpretation of its significance.

\begin{teo}[Zariski-Nagata Theorem \cite{Zar49, Nagata62}] \label{teo:zarnaga}
	If $ R = k[x_0 , ... , x_n] $ is a polynomial ring and $ \p $ is a prime ideal then:
	\begin{equation}\label{eq:zar_nag_teo}
	\p^{(m)} = \bigcap_{\substack{ \mm \in \mm\Spec (R)\\ \p \subset \mm}} \mm ^n
	\end{equation}
\end{teo}

Using the equation \ref{eq:max_pow} we can see that in this case the $ n $-th symbolic power of a prime ideal represents the ideal composed by all the polynomials vanishing on the variety with a multiplicity of $ n $, also indicated with the notation:
\begin{equation}\label{eq:ideal_vanish}
	I^{\left<n\right>} = \{ f \in R \text{ that vanishes on } \mathcal{V}(I) \text{ with multiplicity } n\}
\end{equation}
Also is possible to use this theorem to prove that this property, $ I^{(m)} = I^{\left<n\right>} $, also holds for radical ideal associated to a reduced subscheme in $ \PP ^N $, as shown for example in \cite[Corollary 2.9]{Sid09}.
This is an astonishing result that emphasize a purely geometric significance of the symbolic power, in opposition to the normal one. 

Also for radical ideal of a polynomial ring we have a nicer representation:
\begin{teo}\label{teo:sym_radical}
If $ I $ is a radical ideal in a polynomial ring we have:
\begin{equation}
		I^{(m)} = \bigcap_{\p \in \Ass(R/I) } \p^m
	\end{equation}
\end{teo}



\section{Computation of Symbolic powers}

%\begin{tboxtodo}
%Qui vorrei inserire qualcosa su come calcolare effettivamente le potenze simboliche, mostrando come farlo in Macaulay 2. Penso di usare come fonti i due lavori:
%\begin{itemize}
%\item Topics in commutative algebra:
%Symbolic Powers
%Eloìsa Grifo, 1.5 
%\item \fullcite{Grifo17computation}
%\end{itemize}
%
%Risposta:
%Scrivere solo che sono computabili a macchina e mettere le due referenze
%\end{tboxtodo}
%TODO Actual computation of symbolic powers

The actual computation of the symbolic powers is a difficult problem, because it requires the computation of the primary decomposition of $ I $ and its $ n $-th power (to get the factors $ I^n R_\p \cap R $), but this is an NP-hard problem (See Proposition 2.4 of \cite{Serk02} and Section 3.7 of \cite{Swa17} for more informations). 

However is actually possible to evaluate a set of generators for $ I^{(n)} $ for polynomial rings, in \verb|Macaulay2| (\cite{M2}) there exists the \verb|SymbolicPowers| package, code available at  \href{https://github.com/eloisagrifo/SymbolicPowers}{https://github.com/eloisagrifo/SymbolicPowers}. This package uses several methods to evaluate the results, for example for squarefree monomial ideal (so radical) it uses theorem \ref{teo:sym_radical}. 

More information on the computation of symbolic powers and this package are obtainable on \cite{Grifo18Symb} and \cite{Grifo17computation}. 


Here also a small example of the package use:

\begin{lstlisting}[language = Macaulay2]
i1 : loadPackage "SymbolicPowers"; 

i2 : R = QQ[x,y,z];

i3 : I = ideal(x*y, y*z, z*x);

i4 : symbolicPower(I,3)

                     3 3   2 2    2   2     2 2   3 3   3 3
o4 = monomialIdeal (x y , x y z, x y*z , x*y z , x z , y z )

o4 : MonomialIdeal of R

\end{lstlisting}


\section{Fat Points}

Let's consider now an object of interest, that has particular relations with the symbolic powers.\\
Let $ k $ be a field and $ \PP^N$ the $ N $-th projective space over $ k $, consider now the distinct points \linebreak
$ p_1, ... ,p_k \in  \PP^N$ and some positive integers $ m_1 , ... ,m_k $. If we consider the defining ideals $ I(p_1), ... , I(p_k)  \subset k [\PP^N]$, representing the homogeneous polynomials vanishing on the point (before we have also used the notation $ \mm_{p_i}$ that emphasise their role as maximal ideals) we can define another ideal:
\begin{equation}\label{eq:fat_pt}
	I = \bigcap_{i=1}^k I(p_i)^{m_i} \subset k [\PP^N]
\end{equation}
Since is intersection of homogenous ideals $ I $ is also homogeneous and we can use it to define a $ 0 $ dimensional subscheme $ Z \subset \PP^N $, called \textbf{fat point subscheme} and indicated as 
\[
Z = m_1 p_1 + ... +  m_k p_k
\]
And we will denote $ I $ as $ I(Z) $. This ideal represents the homogeneous polynomials that vanishes on $ p_i $ with multiplicity $ m_i $ for all $ i = 1, ... , k $. The support of the scheme is the set of points $ \{ p_1 , ... , p_k\} $.\\
The symbolic power of $ I(Z) $ has the particular from:
\[
I(Z)^{(m)} = \bigcap_{i=1}^k I(p_i)^{m_im}
\]
and so clearly represents the functions vanishing on $ p_i $ with multiplicity $ mm_i $ for all $ i = 1, ... , k $. 

%TODO Scrivere dove li useremo

%\begin{tboxque}
%Vorrei inserire qualcosa in più, per questo potrebbe servirmi l'articolo \cite{Gimigliano89} ma per ora non lo trovo.
%\end{tboxque}
%
%3) and 4)
%Scrivere solo che solo computabili a macchina e mettere in coda le due referenze
%11 e'estremamente outdate, ho referenze molto piu'aggiornate, ma comunque NON si ineriscono



\section{Relation between normal and symbolic power}

For an ideal $ I $ of a Noetherian ring taking the $ n $-th power of $ I $ is a natural algebraic construction (is the $ n $-th product of $ I $ with itself), we can simply have its generators if we know the one of $ I $, in fact if $ I = \left\langle f_1 , ... ,f_k \right\rangle $:
\[ I^n = \left\langle \xi_1 \cdots \xi_n \,|\, \xi_i \in \{ f_1 , ... ,f_k \} \right\rangle \]
but it has not a clear geometric interpretation, in-fact the knowledge of the generators does not give a lot of information on the primary decomposition. On the other hand we have symbolic power, that inherit a strong geometric significance, for example from Zariski-Nagata theorem, but their generators are tricky to find, so we can pose a natural question:

\begin{que}
What is the relation between the two powers?
\end{que}

An easy result that we can obtain from standard abstract algebra is:

\begin{teo}\label{teo:inv_cont}
	If $ R $ is a Noetherian reduced ring then $ I^r \subseteq I^{(m)}$  if and only if $ r \geq m $
\end{teo}
\begin{proof}
	Since if $ I=0 $ it is obvious we assume it to be non zero.\\
	The first implication is easy, in fact for all $ \p $ we have that $ I^m \subset I^mR_\p \cap R $ since $ I \subset \p $\footnote{So nothing become zero} and $ 1 \not \in \p $.\\
	Suppose now that $ I^r \subseteq I^{(m)}$ and $ r < m $. Consider an associated prime $ \p $, then if we consider the localization $ I_\p $ we have:
	\begin{itemize}
	\item $ (I_\p)^r = (I^r)_\p $ it is obvious
	%TODO Sono la stessa notazione I_\p e IR_\p ??????
	\item $ (I_\p)^m \supset (I^{(m)})_\p $, because $ I^{(m)} \subset I^mR_\p \cap R = I^m_\p \cap R $ and thus passing to the localization (again for the second term) we get the containment.
	\end{itemize}
	So composing this with the containment hypothesis (localized) we get $ (I_\p)^r = (I^r)_\p \subset (I^{(m)})_\p \subset (I_\p)^m $. Since the other inclusion is obvious we have $ I_\p^m= I_\p^r $, considering the intermediary power we have that $ I_\p I_\p^r =I_\p^r $, thus we can use the Nakayama Lemma:
	
	\begin{lem}[Nakayama] \label{lem:naka}
		Given an ideal $ I $ of a commutative ring with unity $ A $ and $ M $ a finitely-generated module over $ A $ with $ IM = M $, then there exists a $ x \in A $ such that $ x \equiv 1 \mod I $ and $ xM = 0 $
	\end{lem}
	%TODO Scolta non è che l'estensione di I in R_\p e l'estensione non sono la stessa cosa?
	In our case the ring is $ R_\p $, the ideal is $ I_\p $ and the module is $ I_\p^{r} $. So since $ R_p $ is a local ring with maximal ideal the localization of $ \p $, that when localized is also the maximal Ideal, and $ I_\p $ is contained in it. Since $ x -1 \in I_\p \subseteq \p_\p $, we cannot have $ x \in \p_\p $, otherwise we would have $ 1 \in \p_\p $, therefore we have that $ x $ is invertible (because $ \p_\p $ is the only maximal ideal) and then $ I_\p^r = 0 $, but since we inherit that $ R_\p $ is a reduced ring this is possible only if $ I_\p = 0$. Since this is true for all the prime ideal (for the non associated one it is obvious since they intersect $ R \setminus I $) and this is a local property (\cite[Proposition 3.8]{AMCD}) we have $ I = 0 $, absurd.
\end{proof} 
Sadly the other direction of the containment isn't that easy, it's an open question and in the last years was largely studied for several classes of ideals.

\chapter{The Containment Problem}
\section*{}
	As stated before the other direction of the containment of symbolic powers into normal powers is an open question in Algebraic Geometry and Commutative Algebra. The general form of the problem is:
	\begin{que}[Containment Problem] \label{que:cont}
		Given a Noetherian Ring $ R $ and an ideal $ I $, for which $ m,r $ positive integers we have the containment:
		$$ I^{(m)} \subset I^r $$
	\end{que}
	For our present knowledge in Algebra Question \ref{que:cont} seems quite general and does not have a unique and simple answer (contrary to the inverse, that has theorem \ref{teo:inv_cont}). Usually we need to specify a particular ring and a particular ideal. Also we consider some precise pairs, like $ 3,2 $, or a subset given by a disequation. 
	\begin{rem}
	If the containment holds for $ m,r $ it does also for all $ m' ,r $ with $ m'\geq m $, since we have $ I^{(m')} \subset \cont{m}{r} $
	\end{rem}
	To better explain this let's see a celebrated result, showed in \cite{HocHun02,EinLazSmi01}:
	\begin{teo}{(Ein-Lazarsfeld-Smith, Hochster-Huneke)}\label{teo:cont:bigh}
	Let $ R $ be a regular ring and $ I $ a non-zero, radical ideal, then if $ h $ is the big height of $ I $ we have that for all $ n \geq 0 $ we have:
	\[ \cont{hn}{n}\]	
	\end{teo} 
	
	To understand this theorem we need two concepts.
	\subsection{Regular ring}
	For a Noetherian local ring $ (R,\mm) $ we say that it is a regular local ring if the minimal number of generators of the maximal ideal is equal to the dimension of $ R $. The name came from a Zariski's result: for an algebraic variety a point $ p $ is non singular (regular) if and only if the ring of germs in $ p $ is regular (\cite{Zar40}). Is possible to see this in a more modern way , in fact for Nakayama's Lemma (is lemma \ref{lem:naka}, but in a different form) is possible to show that $ R $ is regular if and only if $ \dim(\mm / \mm^2 ) = \dim (R) $, and from algebraic geometry $ \mm /\mm^2 $ is the cotangent space of the point corresponding to $ \mm $, so the tangent space has same dimension of the variety if and only if the localization is regular.\\
	In general we say that a Noetherian ring is \textbf{regular} if the localization at every prime ideal is a regular local ring. Also a geometrical interpretation of this definition is that for an affine variety $ V $ its ring of regular functions $ \mathcal{O}_V $ is a regular ring if and only if $ V $ is a non singular variety. 
	
	\subsection{Height of an ideal}
	The \textit{height} of a prime ideal ($\hgt ( \p )$) in a Noetherian Ring $ R $ is the supremum of the lengths $ h $ of prime ideals chains descending from $ \p $:
	\begin{equation}\label{eq:chain}
		\p_0 \subsetneq \p_1 \subsetneq ... \subsetneq \p_{h} = \p
	\end{equation}
	The concept of height is equivalent to the codimension of the ideal $ \p $, that is the (Krull) dimension of the localization $ R_\p $ (looking at the definition it is easy to see that they are the same). Similarly we can define the coheight of $ \p $ as the dimension of the ideal $ \p $ (the supremum of the length of chains ascending from $ \p $). To be more clear we recall that the Krull-dimension of a Ring $ R $ is the supremum of the length of chains of prime ideals:
	\[ \dim (R) = \sup \{ r \,|\, \text{exists a prime ideals chains: } \p_r \subsetneq \p_{r-1} \subsetneq ... \subsetneq \p_0 \}\] 
	And the dimension of an ideal $ I $ is the dimension of the quotient $ R/I $. Moreover, for a $ R $-module $ M $ we define the dimension of $ M $ as the dimension of the annihilator $ \Ann_R(M) $. Observe that for an ideal these two definitions do not coincide (an ideal is also a module), but usually the right calculations is clear from the context. \\
	For a general ideal $ I $ we define the height of $ I $ as the minimum height of its prime ideals (for proposition \ref{prop:minprimes} we can consider only the associated ones) and the \textbf{big height of $ I $} as the maximal height of its associated primes. Also if all the prime assassins as same height these two quantities are equal and we say that the ideal has pure height.


%	\begin{tboxque}
%	Ho visto che alcuni usano il termine \textit{Largest analytic spread in the localization},
%	Dovrebbe essere lo stesso dell'altezza, ma vorrei controllare
%	\end{tboxque}
%
%  domanda 5: non usarlo, usa un solo nomedomanda 5: non usarlo, usa un solo nome
%
	
	An example of a non trivial use of theorem \ref{teo:cont:bigh} on a fat-point subscheme came directly from \cite[2.3]{EinLazSmi01}:
	
	\begin{ex}\label{es:P2points}
	Consider a %reduced\footnote{A scheme is reduced if and only if the associated ideal is radical} come e' scritto ora e'proprio sbagliato \{p_1,\dots ,p_k\} e'reduced (finite sets with its reduced structure) m_1p_1+...+m_kp_k is NOT reduced, unless m_i=1 for all i.
	%Quando dicono reduced intendono l insieme finito, le altre sono le sue potenze simboliche e non sono reduced
	fat point subscheme $ Z = m_1 p_1 + ... +  m_k p_k$ (o more simply a finite set of points) in $ \PP^2 $, since the subscheme has dimension $ 0 $ the ideal has big height $ 2 $, so we have $ \cont{2m}{m} $ for $ I = I(Z) $. Using Theorem \ref{teo:zarnaga} this implies that all $ F $ with multiplicity $ \geq 2m $ (greater or equal than $ 2m_im $ for all the points $ p_i $) stays in $ I(Z)^m $. 
	\end{ex}
	
%\begin{tboxprop}
%	Given that $ k[x_0, ... , x_N] $ is a regular ring (this is a consequence of the Hilbert's syzygy theorem) and that obviously the big hieght of every homogeneous ideal $ I $ is less than the dimension of the ring ($ N $) we have a more geometrically form of theorem \ref{teo:cont:bigh}, that states $ \cont{Nm}{m} $ for all positive $ m $. %TODO Posso farlo per la varietà?
%	Potenzialmente potrei aggiungere anche qualche precisazione da section 19.2 dell'Eisenbud (\cite{Eisenbud95})
%	
%	\end{tboxprop}
%	
%	 
%	\begin{tboxque}
%		Nell'atricolo di Szemberg (\cite{Sze17}) nel teorema 1.3  non richiede che I sia radicale (può essere che assuma essere l'ideale di uno schema ridotto), mi sono perso qualcosa? Mi sembra strano anche perchè lo chiedono nell'articolo di Ein (\cite[Theorem A]{EinLazSmi01}).\\
%		Inoltre non sono certo che la mia motivazione sia giusta (bight $ \leq $ dim(R)) perchè penso ce ne sia una più elegante.
%	\end{tboxque}
	

\section[Constants of relevance]{Constants of relevance for the Containment problem}
To measure the containment property we can some constants associated to the ideal $ I $, one of them is the \textbf{resurgence}, proposed in \cite{Boc09resurgence}, an important article that focuses in the use of some numerical invariant of the ideal to describe the pairs for which the containment problem holds. 
\begin{deff}
	For a proper non-zero ideal $ I $ in a commutative ring $ R $ we define the resurgence of $ I $ as:
	\begin{equation*}
		\rho(I)= \sup \left\lbrace \frac{m}{r} \, | \, I^{(m)} \not \subseteq I^r \right\rbrace 
	\end{equation*}
\end{deff}
Bounding the resurgence of an ideal means finding a constant such that $ \frac{m}{r} > \rho $ implies that $ \cont{m}{r} $ holds.
For example using this quantity we can express the Theorem \ref{teo:cont:bigh} (in a slightly weaker version) as:
\begin{teo}
	For a radical non-zero ideal in a regular ring $ \rho(I) \leq h $ where $ h $ is the big height of $ I $
\end{teo}
In general this is not an optimal bound and since it is difficult to directly evaluate $ \rho(I) $ we can pose the question when the resurgence is strictly less than the big height. 

Another constant, closely related to $ \rho $, is the \textbf{asymptotic resurgence} (introduced in \cite{Guardo2012}), defined as:
\begin{deff}
	For a homogeneous non-zero proper ideal $ I $ of $ k[x_0 , ... , x_n] $ the asymptotic resurgence $ \rho_a(I) $ is:
	\[ \rho_a(I) = \sup \left\lbrace \frac{m}{r} \, | \, I^{(mt)} \not \subseteq I^{rt} \text{ for all } t \gg 0 \right\rbrace \]
\end{deff}
	
\subsection{The Waldschmidt constant}

\begin{deff}\label{def:walds}
Given an homogeneous ideal $ I = \bigoplus_{d>0} I_d$ we can define the Waldschmidt constant as:
\[
\hat{\alpha}(I) = \lim_{m \to \infty} \frac{\alpha (I^{(m)})   }{m}
\]

where $ \alpha(I) $ is the least degree of a generator of $ I $, that is the smallest integer $ d $ such that $ I_d \neq 0 $.
\end{deff}

This constant was introduced for the first time in the 1970' in \cite{Wald77}. And is of particular interest for ideal of fat points. 

\section[Conjectures]{Some conjectures and questions for the Containment Problem}
	A possible question arises from Example \ref{es:P2points}: in this case we know (using $ n=1 $) that $ \cont{4}{2}$, but from several example we know that we can improve the containment to $ \cont{3}{2} $, so naturally arises the question:
	\begin{que}[Huneke] \label{que:32}
	Let I be a saturated ideal of a reduced finite set of points in $ \PP^2 $, does the containment:
	\[ \cont{3}{2} \]
	hold?
	\end{que}
	
%	\begin{tboxprop}
%	Qui vorrei inserire esempi e controesempi, per ora ho trovato e penserei di inserire:\\
%	Controesempi: \textit{Fermat cofiguaration and others} in \cite{DSTG13}. \\
%	Esempi: Star configuartions of points \cite{Har11}.
%	\end{tboxprop}
%	
%	8 non aggiunta
	
	Another good question is if it is possible to improve the result from theorem \ref{teo:cont:bigh}. Since there is no known example for which the bound is optimal a new conjecture have been posed:
	
	\begin{conj}[Harbourne]\label{conj:harb}
		Given a non-zero, proper, homogeneous, radical ideal $ I \subset k[x_0 , ... , x_n] $ with big height $ h $, than for all $ m > 0 $:
		\[
		\cont{hm - h +1}{m}
		\]
	\end{conj}
	
	Trying to solve this problem directly has been shown to be quite difficult, so there are several  sharper version of the Conjecture \ref{conj:harb}, in particular the following one does not request the containment to hold in general, but only asymptotically:
	
	\begin{conj}[Stable Harbourne]\label{conj:stabharb}
			Given a non-zero, proper, homogeneous, radical ideal $ I \subset k[x_0 , ... , x_n] $ with big height $ h $, than for all $ m \gg 0 $:
			\[
			\cont{hm - h +1}{m}
			\]
	\end{conj} 
	
	Another way to modify the Harbourne Conjecture is to use the irrelevant ideal $ \MM=\left\langle x_0 , ... ,x_n \right> $ (also said graded maximal ideal) : 
	
	\begin{conj}[Stable Harbourne-Huneke]\label{conj:stabharbhun}
	Given a non-zero, proper, homogeneous, radical ideal $ I \subset k[x_0 , ... , x_n] $ with big height $ h $, than for all $ m \gg 0 $:
	\begin{itemize}
	\item $ I^{(hm)} \subset \MM^{m(h-1)} I^m$
	\item $ I^{(hm - h +1 )} \subset \MM^{(m-1)(h-1)} I^m $
	\end{itemize}
	\end{conj}
	
	One simple example of why do we use the graded maximal ideal is:
	
	\begin{prop} \label{prop:eulid}
	Given a $ r>0 $ and a non-zero, proper, homogeneous ideal $ I \subset k[x_0 , ... , x_n] $, with $ k $ of characteristic $ 0 $ we have:
	\[ I^{(r+1)}   \subseteq \MM I^{(r)}\]
	\end{prop}
	\begin{proof}
	This is a straight application of Euler identity for homogeneous polynomial:
	\[ (\deg{F})F = \sum_{i=0}^{n} x_i \parder{F}{x_i} \]
	, in fact if $ F \in I^{(r+1)} $ we have $ \parder{F}{x_i} \in I^{(r)} $ for Zariski-Nagata theorem (\ref{teo:zarnaga}) and the thesis follows.
 	\end{proof}
 	

\section{Some results on the Containment problem}

We recall some results on the Containment problem that we will use. First of all a generalization of theorem \ref{teo:cont:bigh}:

\begin{teo}[Theorem 4.4 of \cite{John14}]\label{teo:cont:bighgen}
Let $ R $ be a regular ring containing a field and let $ I $ be an ideal with big height $ h $, for all integers $ n \geq 1 $, $ s \geq 0 $ and $ 0 \leq s_1 \leq  ... \leq s_n $ integers such that $ s = s_1 +  ... + s_n $ we have:
\begin{equation}\label{eq:cont:bighgen}
	I^{(nh + s)} \subseteq \prod_{i=1}^{n} I^{(s_i + 1)}
\end{equation}
\end{teo}

For $ s=0 $ we obtain $ I^{(nh)} \subseteq \prod_{i=1}^{n} I^{(1)} = ( I^{(1)} )^n = I^n$ (theorem \ref{teo:cont:bigh}). Observe that \ref{eq:cont:bighgen} is the other direction of point $ 3. $ in Remarks \ref{rem:symb_basic} and as it happened for the containment problem only one of the two directions is elementary. 

Another results on the containment problem came again from \cite{BocciHrabourne10}, in which it uses the least degree of a generator $\alpha(I)$ and the \textit{Castelnuovo-Mumford regularity}:

\begin{deff} \label{def:reg}
Let $ I \subset R $ be an homogeneous ideal in a graded ring, and let 
\[
0 \to \dots \to F_j \to \dots \to F_0 \to I \to 0
\]
be the minimal free resolution of $ I $ over $ R $, let $ f_j $ be the maximal degree of a generator in a minimal set of generators of  $ F_j $, then the regularity of $ I $ is:
\[ 
 \reg(I) := \max_{j\geq 0} \{f_j - j \}
\]
\end{deff}

Is possible to find more information in \cite[Section 20.5]{Eisenbud95}. 
%\begin{tboxtodo}
%Qui voglio inserire alcune referneze, senza entrare troppo nel dettaglio della regolaità, direi di usare section 20.5 dall'Eisenbud (\cite{Eisenbud95}) 
%\end{tboxtodo}

\begin{lem}[Postulation Criterion 2 from \cite{BocciHrabourne10}] \label{lem:cont:reg}
Let $ I $ be a homogeneous ideal defining a $ 0 $-dimensional sub-scheme in $ \PP^N $, if we have the inequality:
\begin{equation}\label{eq:cont:reg}
r \cdot \reg(I) \leq \alpha(I^{(m)})
\end{equation}
then we have:
\[ \cont{m}{r}\]
\end{lem}


Here there are some sufficient conditions for the Stable Harbourne Conjecture (\ref{conj:stabharb}) to hold, discovered by Eloìsa Grifo in \cite{Grifo20}. 


\begin{teo} \label{teo:cont:grifo0}
Let $ I $ be a radical ideal of big height $ h $ in a regular ring $ R $, suppose that one of this conditions holds:
\begin{enumerate}
\item if exists $ m>0 $ such that $ \cont{hm - h}{m} $
\item if exists $ m>0 $ such that $ \cont{hm - h +1 }{m} $ and for all $ r \geq m $ we have $ \cont{n+h }{ }I^{(n)}$
\item if the resurgence of the ideal satisfy $ \rho(I) < h $
\end{enumerate}
then for $ n \gg 0  $ we have $ \cont{hn - h +1}{n} $
\end{teo}

The first condition is a direct consequence of this theorem, using that $ \cont{f+1}{(f)} $ for all $ f $:

\begin{teo}[Theorem 2.5 of \cite{Grifo20}] \label{teo:cont:grifo1}
Let R be a regular ring containing a field, and let $ I $ be a radical ideal with big height $ h $, if it exists $ m>1 $ with $ \cont{hm -h}{m} $ then for all $ k \geq hm $ we have:
$$ \cont{hk - h}{k} $$
\end{teo}

\begin{proof}
For a $ k\geq hm  $ we can write $ k = hm + t $ with $ t\geq 0 $. Then the idea of this proof is to use theorem \ref{teo:cont:bighgen}, with $ n= h+ t $, $ s = h^2m - h^2 - h$ and 
\begin{equation}\label{eq:defsi}
s_i = \begin{cases}
hm-h-1 & \text{for } i = 1 , ... ,h\\
0 &\text{for }  i =h+1 , ..., h + t
\end{cases}
\end{equation}
thus we have $ hn + s = h(h+t) + h(hm-h-1)= h^2 + ht + h^2m - h^2 - h = h(hm + t) - h = hk - h$. Hence using the theorem:
\begin{equation}\label{eq:teo:grifo1}
I^{ ( hk - h)} = I^{ (hn + s)} \subseteq {( I^{ (mh -h ) } )}^h {I}^t \stackrel{*}{\subseteq} (I^m)^h I^t = I^{ mh +t} = I^k
\end{equation}
Where in $\ast$ we use the hypothesis $ \cont{hm-h}{m} $.
\end{proof}


%\begin{tboxque}
%%Nel teo \ref{teo:cont:bighgen} va bene $ s \in \Z $? o serve $ s\geq 0 $?\\
%Vorrei sapere se fosse interessante inserire qualcosa sulla \textit{reduction to characteristic p}, ho visto che diversi articoli ne fanno riferimento per prendere risultati in caratteristica finita e portarli in caratteristica 0: secondo Lei vale la pena? Nel caso dove mi converrebbe studiare?
%\end{tboxque}


%No
%\begin{tboxtodo}
%Dimostrazioni di altri punti del teorema \ref{teo:cont:grifo0}.\\
%Potrei anche inserire la dimostrazione del teorema \ref{teo:cont:bigh} in caratteristica finita.\\
%\end{tboxtodo}
 	
	


\chapter{Steiner Configuration ideal}
\section*{}

%NO
%\begin{tboxtodo}
%Qui penso di aggiungere una introduzione sugli steiner system e i loro utilizzi.\\		
%Non saprei che testi usare però
%\end{tboxtodo}

A \textbf{Steiner system} $ (V,B) $ of type $ S(t,n,v) $ is an \textit{hypergraph} with $ |V|=v $ and all the elements of $ B $, called blocks, are $ n $-subsets (of $ V $) such that every $ t $-tuple of elements in $ V $ is contained in only one block of $ B $. \\
To be more clear we recall that an hypergraph $ (V,B) $ is a generalization of the normal graph, in which $ V $ is a finite set and $ B $ contains non-empty subset of $ V $ called hyper edges (a normal graph contains only pairs) such that they cover $ V $ ($ \bigcup_{H \in B} H = V $).\\
Geometrically the blocks can be seen as linear subspace in a projective space that contains points in $ V $, in particular this interpretation is useful for \textbf{Steiner triple system}, that are Steiner system with $ t=2 $ and $ n=3 $, also indicated with $ STS(v) $. Later we will use again algebraic geometry, but with a different approach.

\begin{ex} \label{ex:fano1}
	The most known example of Steiner is of type $ STS(7) $ and, up to isomorphism, is the Fano Plane ($ \PP_{\mathbb{F}_2}^3 $). It has as blocks all the lines (hyperplanes):
	\[B := \{\{1, 2, 3\}, \{3, 4, 5\}, \{3, 6, 7\}, \{1, 4, 7\}, \{2, 4, 6\}, \{2, 5, 7\}, \{1, 5, 6\}\}\]
\end{ex}


%\begin{figure}
%\includegraphics[width=10cm,height=10cm,keepaspectratio]{images/fanoplane.png}
%\caption{Fano plane}
%\label{fig:fanoplane}
%\end{figure}


In general the existence of a Steiner system depends on the parameters, for instance a for a Steiner Triple system ($ t=2, n=3 $) we need $ v \equiv 1,3 \mod 6 $. There are not known sufficient existence conditions, but only necessary, for example if it exists a $ S(t,n,v) $ Steiner system we need: 
\[
	|B| = \frac{\binom{v}{t}}{\binom{n}{t}}
\]
This is simply combinatorics, in fact every $ t $-tuple of vertices is contained in only one block and each one of these contains $ \binom{n}{t} $ $ t $-tuples. 

\section[Ideal of Steiner configuration]{An algebraic representation of Steiner systems}

As said before is possible to use algebraic geometry to represents the Steiner systems, in particular the concept of star configuration of points:

%Star configuarations
\begin{deff} \label{def:starconf}
A finite set of points $ Z \subset \PP_k^n $ is a \textbf{star configuration of points} of degree $ d \geq n $ if there exists $ d $ general hyperplanes such that the points of $ Z $ are exactly the ones that are intersection of $ n $ of these hyperplanes.
\end{deff}
By general position we mean that any group of $ n $ hyperplanes intersect in only one point and there is no point belonging to more than $ n $ hyperplanes. It also used the notation $ d $-star to emphasize the degree. 

In our case we consider v-star configurations in $ \PP_k^n $ defined by the general hyperplanes in $ \PP^n $ $ \HH  = \{ H_1 , ... , H_v\}$, with $ H_i $ associated to the linear form $ l_i $ (a linear map from $ \PP_k^n $ to the field of scalars $ k $). Given an $ n $-subset of $ V $\footnote{Since $ V $ is finite we can index it using natural numbers and assume $ V = \{ 1 , ... , v\} $} $ \sigma : = \{ \sigma_1 , ... , \sigma_v\}$ we can associate to it a point of the configuration $ P_{\HH, \sigma }= \cap_{\sigma_i \in \sigma} H_{\sigma_i}$, that has as vanishing ideal $ I_{P_{\HH, \sigma}} = \left< l_{\sigma_1} , ... , l_{\sigma_v} \right\rangle $.\\
 Observe that in this case the vertices are represented by $( n-1 )$-linear space and the blocks by points.\\
 Also we can define:
 \begin{deff}\label{def:gensys}
 Given a finite set $ V $ and a collection of non empty subset $ \mathcal{F} $ we can define, using the previous notation we can define the set of points:
 \begin{equation}\label{eq:X}
 X_{\HH , \mathcal{F}}:= \bigcup_{\sigma \in \mathcal{F} } P_{\HH, \sigma }
 \end{equation}
 and its defining ideal:
 \begin{equation}\label{eq:I}
 I_{X_{\HH , \mathcal{F}}}:= \bigcap_{\sigma \in \mathcal{F} } I_{\HH, \sigma }
 \end{equation}
 \end{deff}
 Please notice that these constructions are more general, so to obtain a Steiner System we assign $ \mathcal{F}=B $, obtaining $ X_{\HH , B} $ and $  I_{X_{\HH , B}} $. 
 We call  $ X_{\HH , B} $ the \textbf{Steiner configuration of points} associated to the Steiner system $ (V , B) $ of type $ S(t , n,v) $ with respect to $ \HH $. \\
 Also we indicate $ C_{(n,v)} $ as the family of all the n-subset of $ V $ and we can construct the \textbf{Complement of a Steiner configuration} of points with respect to $ \HH $ as the scheme $ X_{\HH , C_{(n,v)} \setminus B} $ (said C-Steiner and indicated $ X_C $ too).
Now we obtain some interesting results for this particular scheme. 
\begin{rem}
Since every point in $ \PP_k^n $ has height $ n $ we have that all these ideals has pure height $ n $
\end{rem}
 
 \section{Containment problem for C-Steiner System}
 
 First of all we recall some results from \cite{Bal20Steiner} in particular the Theorem 3.9:
 
 \begin{teo} \label{teo:alphaXC}
Consider a Steiner system $ (V,B) $ of type $ S(t,n,v) $, let $ X_C \subset \PP^n$ be the correspondent C-Steiner configuration and $ I_{X_C} $ its ideal, then:
\begin{enumerate}
\item $ \alpha(I_{X_C}) = v- n $
\item $ \alpha(I_{X_C}^{(q)}) = v- n +q  $ for $ q \in [2,n) $
\item $ \alpha(I_{X_C}^{(m)}) = \alpha(I_{X_C}^{(q)}) + pv $ where $ m=pn + q $ and $ q \in [0,n) $
\end{enumerate}
 \end{teo}
 
 The idea behind the proof of this results is to construct a new simplicial complex $ \Delta_C $ where we can evaluate $\alpha$, thus show that the symbolic powers of the associated ideals $ I_{\Delta_C}^{(m)} $ and $ I_{X_C}^{(m)} $ share the same homological invariants. 
 
 
% puo'scrivere a parole in non piu'di due righe senza simboli, l'idea e'dimostrare che two certain simplicial complex have the same homological invariants
% \begin{tboxprop}
% Vorrei spiegare solo l'idea dietro la dimostrazione, perchè penso che metterla tutta sarebbe eccessivo, in particulare pensavo di dire qualcosa su:\\
% Utilizzo simplicial complex\\
% $ I_{X_C}^{(m)} $ e $ I_{\Delta_C}^{(m)} $ hanno le stesse invarianti omologiche
%  \end{tboxprop}
  
\begin{rem}\label{rem:alphaXC}
We can use the results from theorem \ref{teo:alphaXC} to get some situation in which the containment problem fails, in fact for non-zero, proper, homogeneous ideals $ I $ and $ J $ is straightforward that if $ \alpha(I) < \alpha(J) $ then $ I \not\subseteq J $, in fact $ I_{ \alpha(I) } \neq 0$ but $ J_{ \alpha(I) } = 0 $. 
\end{rem}

\begin{cor}\label{cor:alphaXC}
In the same hypothesis of theorem \ref{teo:alphaXC} we have $ I_{X_C}^{(m)} \not \subseteq I_{X_C}^d $ for any pair $ (m,d) $ such that:
\begin{equation}\label{eq:cor:alphaXC:1}
	m \equiv 1 \mod n \text{ and } d > 1 + \frac{ (m-1)v }{ n(v-n)}
\end{equation}
or 
\begin{equation}\label{eq:cor:alphaXC:2}
	m \not \equiv 1 \mod n \text{ and } d > 1  + \frac{ m - n }{ n } + \frac{ m }{ v - n }
\end{equation}

 In particular if $ v > 2n $ we have $ I_{X_C}^{(n)} \not \subseteq I_{X_C}^2  $
 \end{cor}
  
 \begin{proof}
 Using $ 1. $ from Theorem \ref{teo:alphaXC} and simple algebra we have $ \alpha({I_{X_C}}^d) = d\alpha(I_{X_C}) = d(v- n) $, than from remark \ref{rem:alphaXC} is enough to prove:
 \begin{equation}\label{eq:proof:alphaXC}
 \alpha(I_{X_C}^{(m)}) < d(v-n)
 \end{equation}
 \begin{itemize}
 \item[$m \equiv 1$ : ] we have $ m = pn + 1 $ with $ p $ integer, so using $ 3. $ and $ 1. $ of theorem \ref{teo:alphaXC} for $ q = 1 $ we have $  \alpha(I_{X_C}^{(m)}) \eqb \alpha(I_{X_C} ) + pv \eqb (v-n) + \frac{m-1}{n}v  $. Grouping by the factor $ (n-v) $ and using the second part of \ref{eq:cor:alphaXC:1} we get \ref{eq:proof:alphaXC}. 
 \item[$m \not\equiv 0,1$ : ] we have $ m = pn + q $ with $ q = 0 $ or $ 2 \leq q < n $ so using $ 3. $ and $ 2. $ of theorem \ref{teo:alphaXC} we get  $  \alpha(I_{X_C}^{(m)}) \eqb \alpha(I_{X_C}^{(q)} ) + pv \eqb v - n + q  +pv = (v - n) + m -pn +pv = (1 + p)(v - n) + m $. We can now simply observe that $ p = \frac{m-q}{n} \geq \frac{m-n}{n} $ and then group again by $ v-n $ to use the second part of \ref{eq:cor:alphaXC:2} hence we get \ref{eq:proof:alphaXC}.
 \item[$ m\equiv 0 $ : ] we have $ m=pn $, then \ref{eq:cor:alphaXC:2} became with simple algebra $ d > \frac{pv}{v-n} $, and hence for $ 3. $ of theorem \ref{teo:alphaXC} $  \alpha(I_{X_C}^{(m)})= pv < d(v-n)$, that satisfy \ref{eq:proof:alphaXC}. 
 \end{itemize}
 In particular for $ m=n $ and $ v > 2n $ we have 
 \[ 1 +\frac{ m - n }{ n } + \frac{ m }{ v - n } = 1 + \frac{ n }{ v - n }= \frac{ v }{ v - n } < 2 \]
 and hence $ I_{X_C}^{(n)} \not \subseteq I_{X_C}^2  $
 \end{proof}
 
 \begin{ex}\label{ex:fano2}
 Consider now Steiner system of type $ S(2,3,7) $ as in example \ref{ex:fano1}, given $ \HH = \{ H_1 , ... , H_7 \} $ a collection of 7 hyperplanes in general position in $ \PP^3 $ we can use them to construct the C-Steiner configuration $ X_{\HH , C_{(3,7)} \setminus B} $ , it has:
 \[
 \binom{v}{n} - |B| = \binom{v}{n} -  \frac{\binom{v}{t}}{\binom{n}{t}} = \binom{7}{3} -  \frac{\binom{7}{2}}{\binom{3}{2}} = 28
 \]
 points in $ \PP^3 $. Its defining ideal is  $ I_{X_C} $. \\
 Using the corollary \ref{cor:alphaXC} we get $ I_{X_C}^{(3)}  \not \subseteq I_{X_C}^2$, that is a new counterexample for question \ref{que:32}. 
 \end{ex}
 
Other useful results from \cite{Bal20Steiner} are the calculation of some homological invariants of $ I_{X_C}$, like the regularity and the largest degree in a minimal homogeneous set of generators of the ideal, i.e. the integer $ \omega (I_{X_	C}) $ such that given a minimal set of generators $\{ f_1 ,... , f_k\} $ of $ I_{X_	C} $ we are granted to have $ \deg (f_i) \leq \omega (I_{X_	C}) $ for all $ i $ and the bound is reached by at least one element. %TODO Dovrei dimostrare che è definito bene?

\begin{prop}[Corollary 4.2 and 4.7 in \cite{Bal20Steiner}]
\label{prop:largdeg}
For a Steiner system of type $ S(t,n,v) $ we have:
\begin{equation}\label{eq:largdeg}
\omega (I_{X_	C}) = 
\begin{cases}
\alpha (I_{X_	C})=v-n     & \text{ if } t = n-1\\
\reg(I_{X_C}) & \text{ if } t < n-1
\end{cases}
\end{equation}
with  \begin{equation}\label{eq:reqXC}
\reg(I_{X_C}) =\alpha (I_{X_	C})+1= v-n +1
\end{equation}
\end{prop}

In particular we will use this proposition in combination with the lemma:
\begin{lem}\label{lem:cont_om}
Let $ I,J $ be homogeneous ideals in a Polynomial Ring $ R $, if we have both $ I \subseteq J $ and $ \alpha(I) - \omega (J) \geq k >0 $ then we have:
\[ I \subseteq \MM^k J\]
\end{lem}
\begin{proof}
Consider a minimal set of generators $ \left\langle \xi_1 , ... ,\xi_r \right\rangle = J $, with degree bounded by $ \omega(J) $, then for an element $ f \in I $ (without loss og generality we assume it homogenous) we have $ f = \sum g_i \xi_i  $, thus we have for all $ i $ with $ g_i\neq 0 $:
\[ \alpha(I) \leq \deg (f) = \deg(g_i \xi_i ) = \deg(g_i) + \deg(\xi_i) \leq  \deg(g_i) + \omega(J) \]
therefore we have:
\[ \deg(g_i)  \geq\alpha(I) - \omega (J) \geq k \]
so $ g_i \in \MM^k $ and the thesis follows.
\end{proof}


Here we prove one of the main result of the article \cite{Bal21Steiner}:
\begin{teo}\label{teo:cont:csteiner}

Let $ I_{X_C} \subset k[x_0 , ... , x_n]$ be the defining ideal of a Complement of a Steiner configuration of points in $ \PP^n_k $, then it satisfies:
\begin{enumerate}
\item Stable Harbourne-Huneke Conjecture (\ref{conj:stabharbhun}) (Proof of the second point only for the case of $ t = n-1 $)
\item Stable Harbourne Conjecture (\ref{conj:stabharb})
\end{enumerate}
\end{teo}

\begin{proof}
\begin{enumerate}
\item Let $ I = I_{X_C}$ be the ideal defining the Complement of a Steiner configuration of points of type $ S(t,n,v) $ in $ \PP_k^n $, we will use the lemma \ref{lem:cont_om} in combination with proposition \ref{prop:largdeg} and theorem \ref{teo:alphaXC} to evaluate $\omega$ and $\alpha$:
	\begin{itemize}
	\item Using theorem \ref{teo:cont:bigh} with $ h=n $ we have $ \cont{hr}{r} $ for all $ r>0 $, also we have $ \alpha(I^{(nr)}) = rv $ and $ \omega(I^r)=r\omega(I)  \leq r(v-n+1) $, so $  \alpha(I^{(nr)}) - \omega(I^r) \geq rv - rv + r(n-1) = r(n-1) $, thus for all $ r>0 $:
	\[ 
	\mcont{nr}{r(n-1)}{r}
	\]
	\item We have $\alpha(I^{(n(r-1))}) = (r-1)v $ and $ \reg(I) = v-n+1 $ (Proposition \ref{prop:largdeg} equation \ref{eq:reqXC}), thus for $ r \gg 0 $ we have:
	\[ \alpha(I^{(n(r-1))}) = (r-1)v \geq r(v-n+1) = r \reg(I)\]
	so for the Postulation Criterion (Lemma \ref{lem:cont:reg}) we have the containment $ \cont{n(r-1)}{r} $. Now we observe that assuming that $ t=n-1 $:
\begin{equation} \label{eq:alpha?}
	\alpha(I^{(n(r-1))}) - \omega (I^r) = (r-1)v - r(v-n)=rn -v
\end{equation}
 And so we have $ \mcont{n(r-1)}{rn-v}{r} $, thus we have:
 \begin{multline*}
 I^{(n(r-1)+1)} \stackrel{*}{\subseteq} \mathcal{M} I^{(n(r-1))} \subseteq \mathcal{M}^{r n-v+1} I^{r} =\mathcal{M}^{r n-v-n+n+1} I^{r}=\\
 =\mathcal{M}^{r n-n-(v-n)+1} I^{r} 
  \stackrel{**}{\subseteq} \mathcal{M}^{r n-n-r+1} I^{r}=\mathcal{M}^{(r-1)(n-1)} I^{r}
 \end{multline*}
 Where in $ \ast $ we have used proposition \ref{prop:eulid} and in $ \ast \ast $ that $ r \gg 0 $ to have $ r \geq v-n $.
%	
%	\begin{tboxque}
%	Non mi è chiaro perchè in \ref{eq:alpha?} basti fare l'uguaglianza con $ \alpha $ senza usare $\omega$, dato che l'ugualianza c'è solo per $ t = n-1 $. Ho provato un po' a correggere ma non sono riuscito\\
%	Inoltre come si può vedere ho unito proposizione 3.4 e teorema 3.5 di \cite{Bal21Steiner}, tirandone fuori il lemma e il teorema perchè mi è venuto più semplice lavorare così, spero vada bene
%	\end{tboxque}
	
	\end{itemize}
\item Using theorem \ref{teo:cont:grifo0}, condition \textit{1.} , and having for some $ r $ the containment $ \cont{n(r-1)}{r} $ we get for $ r \gg 0 $ the containment $ \cont{hn - h +1}{n} $
\end{enumerate}
\end{proof}


 

\chapter{Colouring of an hypergraph and containment}

In this chapter we focus in an interesting relation between the colourability of an hypergraph and  the failure of the containment problem of its cover ideal. Then we apply these results in the case of a Steiner System. These results were proposed for the first time in the section 4 of \cite{Bal21Steiner}.
 
\section{Colouring definitions}

We can see two different definition of $ m $-colouring, the first one is more descriptive and use the intuitive notion of surjective map, focusing on the role of the vertices: 
\begin{deff}\label{def:colouring1}
Let $ C $ be the set of possible colours, a colouring of the hypergraph $ H = (V,E) $ is a surjective map $ c : V \to C $. \\
In the case of which $ C $ is finite we assume $ C = \{ 1 , ... , m\} $ and we say that a colouring $ c : V \to C $ is a proper $ m $-colouring if for every hyperedge $ \beta \in B $ we have at least two vertices of different colours. 
\end{deff}

The second one is due to the fact that we can see a colouring as a partition of the vertices (in fact is well known that a surjective map partitions the domain using the fibres):

\begin{deff}
An $ m $-colouring of the hypergraph $ H = (V,E) $ is a partition in $ m $ subset of $ V = U_1 \sqcup ... \sqcup U_m $ such that for every edge $ \beta \in E $ we have $ \beta \not \subseteq U_i $ for all $ i = 1, ... , m $
\end{deff}
 
We introduce the \textbf{chromatic number} of the hypergraph $ H $, indicated $ \chi(H) $, that is the minimum $ m $ such that $ H $ has an $ m $-colouring.

\begin{deff}
A hypergraph $ H $ is $ m $-colourable if there exists a proper $ m $-colouring
\end{deff}

If the hypergraph has less than $ m $ vertices the condition of $ m $-colourability is equivalent to the inequality $ \chi(H) \leq m $, in fact if we have a $ c $-colouring $ V = U_1 \sqcup ... \sqcup U_c $ with $ v < c $ we can assume $ U_1 $ to have more than $ 2 $ elements and get a new partition: $ V = \{x_1\} \sqcup U_1 \setminus \{x_1\} \sqcup ... \sqcup U_m $. Thus $ H $ is $ (c+1) $-colourable and we can end by induction.

\begin{deff}
We say that a hypergraph $ H $ is m-chromatic if it is $ m $-colourable, but no $ (m-1) $-colourable (i.e. $\chi(H) = m  $)
\end{deff}

We see now a stronger definition, that also require at every block to have vertices of every colour:

\begin{deff} \label{def:cover}
A hypergraph $ H = (V,E) $ is said \textit{$ c $-coverable} if there exists a partition in $ c $ subset of $ V = U_1 \sqcup ... \sqcup U_m $ such that every $ U_i $ is a \textbf{vertex cover}, which means  that for all $ \beta \in E $ we have $ \beta \cap U_i \neq \emptyset $ ($ U_i $ meets all the edges). 
\end{deff}

\section{Edge between algebra and colouring theory}


This concepts where introduced for graph in \cite{Villa90} and for hypergraph in \cite{Ha08}.

\begin{deff}\label{def:edgecoverideal}
Given a hypergraph $ H = (V,E) $, let $ V = \{ x_1 , ... , x_v \} $ and $ k $ be a field, thus we can identify the vertices as variables in $ k[x_1 , ... , x_v] $ (also indicated $ k[V] $), then we can define the \textbf{edge ideal}:
\begin{equation}\label{eq:edgeideal}
I(H) := \left\langle x_{i_1} \cdots x_{i_r} \,|\, \{ x_{i_1} , ... , x_{i_r}\} \in E \right\rangle 
\end{equation}
and the \textbf{cover ideal} as:
\begin{equation}\label{eq:coverideal}
J(H) := \left\langle x_{j_1} \cdots x_{j_r} \,|\, \{ x_{j_1} , ... , x_{j_r}\} \text{ is a vertex cover of } H \right\rangle 
\end{equation}
\end{deff}

I recall that a \textit{simple hypergraph} is an hypergraph without loops (edges with a single element) and without repeated edges (i.e. if $ \beta, \gamma \in E$ and $ \beta \subseteq \gamma $ then $ \beta = \gamma $). 


The notion of edge ideal is of relevance for several reasons, for example there is a natural bijection between simple hypergraph over $ V $ and squarefree monomial ideal of $ k[V] $:
\begin{align*} \label{eq:bij}
\left\{\begin{array}{c}
\text { simple hypergraphs } \\
H=(V, E)
\end{array}\right\}
&  \leftrightarrow\left 
\{\begin{array}{c}
\text { squarefree monomial } \\
\text { ideals } I \subseteq k\left[V \right]
\end{array}\right\} \\
(\beta = \{ x_{i_1} , ... , x_{i_r}\} \in E )
&  \leftrightarrow  
(x_{i_1} \cdots x_{i_r} \in I)
\end{align*}


It is possible to give a different definition of cover ideal that focuses more on its primary decomposition. Indeed we can define for every $ \beta = \{ x_{i_1} , ... , x_{i_r}\} \in E  $ the ideal:
\begin{equation}\label{eq:prime}
\p_\beta = \left\langle  x_{i_1} , ... , x_{i_r} \right\rangle \subseteq k[V]
\end{equation}
then it is obvious that for every squarefree monomial element $ x_{i_1} \cdots x_{i_r} $ it stays in $ J(H) $ if and only if it stays in $ \p_\beta $ for all $ \beta \in E $ (because this means it intersect every edge), so we can also characterize: 
\begin{equation}\label{eq:coverideal2}
J(H) = \bigcap_{\beta \in E} \p_\beta
\end{equation}

%\begin{tboxtodo}
%Spiegare meglio perchè sono primi associati, è ovvio che i primi associati debbano avere la forma $  \left\langle  x_{i_1} , ... , x_{i_r} \right\rangle $, ma dovrebbe essere valido anche per potenze di ideali monomiali (monomial ideals), ma non ho idee di come mostrarlo
%\end{tboxtodo}
%
%For simple hypergraph this primes are minimal 



Is possible to exploit the algebraic structure of the cover ideal to get information on the colouration of the hypergraph (and vice versa), for example in \cite{Fran10Colourings} the authors discovered that using the element $  \mm_V = x_1 \cdots x_v  $ is possible to evaluate the chromatic number:

\begin{teo}[Theorem 3.2 of \cite{Fran10Colourings}] \label{teo:col:chi}
Let $ H = (V,E) $ be a simple hypergraph on $ V = \{ x_1 , ... , x_v\} $, then for all $ d >0 $ we have $ \mm_V^{d-1} \in J(H)^d $ if and only if $ \chi(H) \leq d $, thus we can characterize:
\begin{equation}\label{eq:col:chi}
 \chi(H)   = \min \{ d \,|\, \mm_V^{d-1} \in J(H)^d\}
 \end{equation}
\end{teo}



For a hypergraph $ H = (V,B) $ we define $ \tau (H) $ as the smallest cardinality for a set $ \min_{\beta \in B} \{ |\beta |\}$, we can use this number to show a link, introduced in \cite{Bal21Steiner}, between the colouring properties of $ H $ and when the Containment Problem fail for the cover ideal:

\begin{teo}[Theorem 4.8 of \cite{Bal21Steiner}] \label{teo:col:cont}
Let $ H = (V,B) $ be a hypergraph, if $ H $ is not $ d $-coverable then $ J(H)^{(\tau(H))} \not \subseteq J(H)^d $
\end{teo}

\begin{proof}
Let $\tau = \tau (H)$ and $ J = J(H) $, then we have:\\
since for each $ \beta \in B $ the ideal $ \p_\beta $ has $ |\beta | \geq \tau $ elements, we have $ \mm_V \in \p_\beta^\tau $ (it is enough to use $\tau$ of the $ |b| $ variables and complete with the missing one). Then, since the cover ideal is an intersection of prime ideals and so is radical, using the theorem \ref{teo:sym_radical} we get $ \mm_V \in J^{(\tau)} $.\\
Suppose now that $ \mm_V \in J^d $, then there exists $ m_1 , ... , m_d \in J $ with $ \mm_V = m_1 \cdots m_d $. We want to use them to construct a partition: $ U_i := \{ x_u \in V \,|\, x_u \text{ divides } m_i\} $. $ U_1 , ... ,U_d $ is a partition because $ \mm_V $ is squarefree (so no repetition) and it is divided by all the vertices of $ V $. Since for all $ i $  $ m_i $ is a squarefree monomial in $ J $, by using the definition $ \ref{eq:coverideal} $ of cover ideal we have that $ U_i $ must be a vertex cover, and then we have a partition that satisfies definition \ref{def:cover} of $ d $-coverable, absurd. 
\end{proof}

We can use this theorem to prove the failure of Question \ref{que:32} for cover ideals of Steiner Triple System $ STS(v) $, in fact it is enough to prove that 
\begin{prop}[Proposition 4.9 of \cite{Bal21Steiner}]
If $ v>3 $ and $ S = (V,B) $ is a Steiner Triple System $ S(2,3,v) $, then $ J(S)^{(3)}  \not \subseteq J(S)^2$
\end{prop}

%\begin{proof}
%Obviously $ \tau(S)=n=3 $, so we need only to prove that $ S $ is not $ 2 $-coverable. Assume that there exists a partition of vertex covers $ U_1 \sqcup U_2 = V $, so for all triple $ \{ i , j, k\} \in B $ we have $ \{ i , j, k\} \cap U_1 \neq \emptyset $ and $ \{ i , j, k\} \cap U_2 \neq \emptyset $, this means that for every edges there are exactly $ 2 $ colours. This characteristic is called $ 2 $-bicolourable, that \textbf{???}. 
%\begin{tboxque}
%Non so bene perchè dovrebbe essere assurdo e purtroppo non trovo l'articolo \textit{Steiner triple systems and their chromatic number, Rosa A.}
%\end{tboxque}
%\end{proof}



Another possible failure of the containment can be deducted by the chromatic number:

\begin{teo} \label{teo:borin1}
	Consider a simple hypergraph $H = (V,B)$, if we indicate $\tau = \tau (H)$ and the cover ideal $ J = J(H)$,  then for all $ q < \chi(H)$ we have: 
	\begin{enumerate}
	\item $ \mm_V^{q-1} \in J(H)^{( \tau (q-1) )} $
	\item $ J^{( \tau(q-1) )} \not \subseteq J^q $
	\end{enumerate}
\end{teo}


\begin{proof}
	From theorem \ref{teo:col:chi} we have $\mm_V^{q-1} \not \in J^q$, therefore we only need to prove \textit{1.} to fail the containment (\textit{2.}). For this we use as in \ref{teo:col:cont} theorem \ref{teo:sym_radical}:\\
	For every $\beta \in B $ we know that it has at least $	\tau$ elements, without loss of generality we can assume that they are $x_1 , ... , x_\tau$, and so we have $x_1^{q-1} \cdots  x_\tau^{q-1} \in \p_\beta^{\tau(q -1)} $ thus completing with the other elements we get $\mm_V^{q-1} \in \p_\beta^{\tau(q -1)} $. 
\end{proof} 


\backmatter
\printbibliography
\end{document}
